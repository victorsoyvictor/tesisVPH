
\chapter*{Resumen en Espa\~{n}ol}
El Pa\'{\i}s Vasco es una regi\'on (Comunidad Aut\'onoma) situada en el norte de Espa\~{n}a donde, desde los a\~{n}os 60, existe una organizaci\'on llamada ETA que quiere alcanzar sus objetivos pol\'{\i}ticos mediante medios violentos. La poblaci\'on vasca est\'a dividida, principalmente, entre los que apoyan o re\-cha\-zan los objetivos pol\'{\i}ticos de ETA y sus medios violentos. Adem\'as, la presi\'on de ETA y de los grupos que la apoyan sobre la poblaci\'on vasca, est\'a presente en la vida cotidiana. 

Teniendo en cuenta la situaci\'on vasca, en esta tesis doctoral estamos interesados en c\'omo los diferentes grupos definidos por su actitud hacia ETA evolucionan con el tiempo, fij\'andonos especialmente en dos aspectos:

\begin{itemize}
\item averiguar cuales son los eventos m\'as relevantes que influyen en los cambios de la evoluci\'on sobre la actitud hacia ETA,
\item teniendo en cuenta la relaci\'on entre los grupos que apoyan a ETA y la fuente de sus activistas, predecir la evoluci\'on de la actitud hacia ETA en el futuro pr\'oximo para saber si el n\'umero de miembros de los grupos que apoyan a ETA disminuyen y, como consecuencia, ETA tambi\'en.
\end{itemize}

Para conseguirlo, por una parte, utilizaremos datos electorales y del Eus\-kobar\'ometro (el Euskobar\'ometro es una encuesta sociol\'ogica realizada en el Pa\'{\i}s Vasco), y as\'{\i}, construiremos modelos matem\'aticos asumiendo como hip\'otesis que el cambio de actitud, ideolog\'{\i}a u opini\'on puede ser socialmente transmitido. Por tanto, podremos utilizar t\'ecnicas cl\'asicas de epidemiolog\'{\i}a para la construcci\'on y estudio de dichos modelos.

Por otra parte, no debemos olvidar que queremos estudiar un problema del \'area de las Ciencias Sociales donde los datos provienen de encuestas, con lo que contienen un error y una incertidumbre inherentes. As\'{\i} pues, durante el desarrollo de esta memoria se har\'a necesario utilizar t\'ecnicas para tratar la incertidumbre en los modelos que iremos presentando. De hecho, en cada nueva t\'ecnica que usemos, intentaremos evitar los inconvenientes que aparec\'{\i}an en la t\'ecnica anterior. 

La estructura de esta memoria es la siguiente. En el cap\'{\i}tulo \ref{CAPINTRO} introduciremos el problema a estudiar y haremos un repaso hist\'orico del trabajo realizado.

En el cap\'{\i}tulo \ref{ETA}, resumiremos los hechos m\'as importantes en la historia de ETA que consideramos relevantes para el adecuado desarrollo de la presente tesis.
 
Presentaremos un primer modelo en el cap\'{\i}tulo \ref{paper1}. En \'el, dividiremos la poblaci\'on vasca dependiendo del partido pol\'{\i}tico al que votan y luego clasificaremos los partidos pol\'{\i}ticos respecto a su opini\'on sobre la idea "independencia de Espa\~{n}a", uno de los principales objetivos de ETA. De esta forma, utilizando datos de las elecciones generales al Parlamento Espa\~{n}ol, construiremos un modelo de tipo epidemiol\'ogico y usaremos la t\'ecnica del Muestreo del Hipercubo Latino para predecir con incertidumbre en futuras fechas electorales, la din\'amica de la poblaci\'on vasca respecto de la idea "independencia de Espa\~{n}a". 

En el cap\'{\i}tulo \ref{paper2}, utilizamos datos del Euskobar\'ometro sobre "la actitud de la poblaci\'on hacia ETA" para construir un modelo que nos permita averiguar si la "Ley de Partidos Pol\'{\i}ticos" (LPP) aprobada en junio de 2002 tuvo alg\'un efecto sobre la actitud de los vascos hacia ETA. Aplicaremos una t\'ecnica llamada "bootstrapping" para saber si la diferencia entre la predicci\'on de modelo y los datos del Euskobar\'ometro tras la LPP es significativa y cuantificar dicha diferencia. En este caso, la t\'ecnica de bootstrapping es la que nos permitir\'a estudiar la incertidumbre. 

Finalmente, en el cap\'{\i}tulo \ref{paper3}, utilizando el mismo modelo que en el cap\'{\i}tulo \ref{paper2} y datos del Euskobar\'ometro referentes a la actitud de la poblaci\'on hacia ETA desde mayo de 2005, predeciremos con incertidumbre la din\'amica de evoluci\'on de los diferentes grupos mediante una banda de confianza del mo\-de\-lo en los pr\'oximos a\~{n}os. Esto lo conseguiremos introduciendo una nueva t\'ecnica computacional para tratar la incertidumbre en el modelo.

\chapter*{Resum en Valenci\`a}
El Pa\'{\i}s Basc \'es una regi\'o (Comunitat Aut\`onoma) situada al nord d'Espanya on, des dels anys 60, hi ha una organitzaci\'o anomenada ETA que vol aconseguir els seus objectius pol\'{\i}tics per mitjans violents. La poblaci\'o basca est\`a dividida, principalment, entre els que recolzen o rebutgen els objectius pol\'{\i}tics d'ETA i els seus mitjans violents. A m\'es, la pressi\'o d'ETA i dels grups que la recolzen sobre la poblaci\'o basca, estan presents en la vida quotidiana.

Tenint en compte la situaci\'o basca, en esta tesi doctoral estem interessats en com els diferents grups definits per la seua actitud cap a ETA evolucionen amb el temps, fixant-nos especialment en dos aspectes:

\begin{itemize}
\item esbrinar quals s\'on els esdeveniments m\'es rellevants que influ\"{i}xen en els canvis de l'evoluci\'o sobre l'actitud cap a ETA,
\item tenint en compte la relaci\'o entre els grups que recolzen a ETA i la font dels seus activistes, predir l'evoluci\'o de l'actitud cap a ETA en el futur pr\`oxim per a saber si el nombre de membres dels grups que recolzen a ETA disminu\"{i}xen i, com a conseq\"{u}\`encia, ETA tamb\'e.
\end{itemize}

Per a aconseguir-ho, d'una banda, utilitzarem dades electorals i de l'Eusko\-bar\'ometro (l'Euskobar\'ometro \'es una enquesta sociol\`ogica realitzada en el Pa\'{\i}s Basc), i aix\'{\i}, construirem models matem\`atics assumint com a hip\`otesi que el canvi d'actitud, ideologia o opini\'o pot ser socialment transm\'es. Per tant, podrem utilitzar t\`ecniques cl\`assiques d'epidemiologia per a la cons\-trucci\'o dels anomenats models.

D'altra banda, no hem d'oblidar que volem estudiar un problema de l'\`area de les Ci\`encies Socials on les dades provenen d'enquestes, amb la qual cosa contenen un error i una incertesa inherents. Aix\'{\i}, durant el desenrotllament d'esta mem\`oria es far\`a necessari utilitzar t\`ecniques per a tractar la incertesa en els models que anirem presentant. De fet, en cada nova t\`ecnica que usem, intentarem evitar els inconvenients que apareixien en la t\`ecnica anterior.

L'estructura d'esta mem\`oria \'es la seg\"{u}ent. En el cap\'{\i}tol \ref{CAPINTRO} introduirem el problema a estudiar i farem un rep\`as hist\`oric del treball realitzat.

En el cap\'{\i}tol \ref{ETA}, resumirem els fets m\'es importants en la hist\`oria d'ETA que considerem rellevants per a l'adequat desenrotllament de la present tesi.

Presentarem un primer model en el cap\'{\i}tol \ref{paper1}. En ell, dividirem la poblaci\'o basca depenent del partit pol\'{\i}tic a qu\`e voten i despr\'es classificarem els partits pol\'{\i}tics respecte a la seua opini\'o sobre la idea "independ\`encia d'Espanya", un dels principals objectius d'ETA. D'esta manera, utilitzant dades de les eleccions generals al Parlament Espanyol, construirem un model de tipus epidemiol\`ogic i usarem la t\`ecnica de l'Hipercub Llat\'{\i} per a predir amb incertesa en futures dates electorals, la din\`amica de la poblaci\'o basca respecte de la idea "independ\`encia d'Espanya".

En el cap\'{\i}tol \ref{paper2}, utilitzem dades de l'Euskobar\'ometro sobre "l'actitud de la poblaci\'o cap a ETA" per a construir un model que ens permeta esbrinar si la "Llei de Partits Pol\'{\i}tics" (LPP) aprovada al juny de 2002 va tindre algun efecte sobre l'actitud dels bascos cap a ETA. Aplicarem una t\`ecnica anomenada "bootstrapping" per a saber si la difer\`encia entre la predicci\'o del model i les dades de l'Euskobar\'ometro despr\'es de la LPP \'es significativa i quantificar eixa difer\`encia. En este cas, la t\`ecnica de bootstrapping \'es la que ens permetr\`a estudiar la incertesa.

Finalment, en el cap\'{\i}tol \ref{paper3}, utilitzant el mateix model que en el cap\'{\i}tol \ref{paper2} i dades de l'Euskobar\'ometro referents a l'actitud de la poblaci\'o cap a ETA des de maig de 2005, predirem amb incertesa la din\`amica d'evoluci\'o dels diferents grups per mitj\`a d'una banda de confian\c ca del model en els pr\`oxims anys. Per a fer a\c c\`o, introduirem una nova t\`ecnica computacional per a tractar la incertesa en el model.


\chapter*{Abstract in English}
The Human Papillomavirus (HPV) is the direct cause of more than half million new cases of cervical cancer, the second most common malignancy among women and a leading cause of cancer death worldwide. It also causes anogenital warts. It is estimated that the probability of transmission of HPV is 40-50\% per contact. The networks of sexual contacts in human populations are crucial to the spread of sexually transmitted diseases (STDs).

Taking into account the aforementioned information, we have built an application that runs in a distributed environment that allow us to study in an efficient way a HPV dynamic transmission model in a lifetime sexual partner networks and its control. Therefore, we are able to simulate our model and apply  vaccination campaigns in order to get conclusions concerning the best strategies. This information can be useful for Health Care, policy makers and pharmacological industries to save time and money.

\begin{itemize}
\item we need a reliable model
of the sexual network substrate in which the pandemic builds up.
\item spread the HPV over the network and simulate the time passing.
\item calibrate the model by adjusting the unknown parameters using an optimization algorithm.
\item apply vaccination campaigns and discuss best strategies.
\end{itemize}

The structure of this PhD dissertation is as follows. In Chapter \ref{CAPINTRO} we introduce the problem to be studied.......... TODO

%In Chapter \ref{ETA} we summarise the main facts in the history of ETA that we consider relevant to the proper development of the present dissertation.

%A first model is presented in Chapter \ref{paper1}. Here, we divide the population of the Basque Country depending on the political party they vote and classify the political parties respect to their opinion on the idea of "independence from Spain", one of the main goals of ETA. Thus, with data of general elections, we build a type-epidemiological model and use the Latin Hypercube Sampling technique to predict with uncertainty over the next election dates, the dynamics of the population respect to the idea of "independence from Spain".

%In the Chapter \ref{paper2}, Euskobarometro data about the "population attitude towards ETA" are used to build a model to find out if the "Law of Political Parties" (LPP) passed in Jun 2002 had effect on the attitude towards ETA of the Basque Population. We use a bootstrapping technique to know if the differences between the model prediction and Euskobarometro data after LPP are significative and quantify these differences. In this case, bootstrapping is the technique that allows us to deal with the model uncertainty. 

%In the Chapter \ref{paper3}, using the same model as in the Chapter \ref{paper2} and Euskobarometro data about the population attitude towards ETA since May 2005, we predict with uncertainty the evolution dynamics of the groups providing a model confidence band prediction over the next few years. To do that we introduce a new computational technique to deal with the model uncertainty.

