
\chapter*{Abstract}
Sexually Transmitted Diseases (STD) have been a major public health threat for a long time in human history. Modern concerns about STD began with the pandemic of syphilis which spread over Europe in the early sixteenth century. 

The Human Papillomavirus (HPV) is the direct cause of more than half million new cases of cervical cancer, the second most common malignancy among women and a leading cause of cancer death worldwide. It also causes anogenital warts. It is estimated that the probability of transmission of HPV is 40-50\% per contact. The networks of sexual contacts in human populations are crucial to the spread of sexually transmitted diseases (STDs).

Working in large networks applied to epidemiological-type models has led us to design a simple but effective computed distributed environment to perform a large amount of model simulations in a reasonable time in order to study the behavior of these models and to calibrate them. Finding the model parameters that best fit the available data in the designed distributed computing environment becomes a challenge and it is necessary to implement reliable algorithms for model calibration.

We are able to simulate our model and carry out vaccination campaigns in order to get conclusions concerning the best strategies. This information can be useful for Health Care, policy makers and pharmacological industries to save time and money.


\chapter*{Resum en Valenci\`a}
