
\chapter{Euskadi Ta Askatasuna (ETA)} \label{ETA}

ETA defines itself as "(...) a Basque socialist revolutionary organization for national liberation" \cite{ETA, ceaseFire, wETA}. One of its core demands is the creation of a Basque State, which would encompass the current three Basque Provinces and Navarra, in Spain, and three more French provinces. According to the Spanish Ministry of Internal Affairs, ETA has committed 829 murderers since 1968 \cite{MI}, although this figure is under debate \cite{Buesa}. 

After 50 years of activity, ETA has been proscribed as terrorist organization by the Spanish authorities, the European Union \cite{PC}, the USA and the United Nations. The long duration of ETA has been studied in \cite{Barros3}.

Batasuna was a left-wing nationalist party and, as the Spanish Supreme Court stated, it was the ETA's political wing \cite{sentencia}.

In June 2002, the "Law of Political Parties" (LPP) was passed in the Spanish Parliament and its goals were "(...) to guarantee the democratic system and citizen's essential freedoms, by preventing some political parties from threatening democracy, justify racism and xenophobia or give political support to terrorist organizations" \cite{LP}. As a consequence of this law, in August 2002, the suspension of the activities of the party Batasuna and the closing of its headquarters was decreed. However, the fact is that Batasuna persisted in political activities banned by the LPP. That circumstance led the Supreme Court to outlaw that organization in March 2003, which implied the eventual cessation of all its activities and the confiscation of its possessions \cite{sentencia}. The organization to conduct any political meeting or propaganda activities was also specifically prohibited \cite{auto}. In June 2003, Batasuna and other related parties were included, as a part of ETA, in the list of terrorist organizations in the European Union \cite{PC}.

The LPP meant a substantial change in the anti-terrorist policy in Spain. In practical terms, once Batasuna was outlawed, this party could not present candidates to elections and consequently, ETA and its environment were not going to be supported  from political institutions nor funded by public budgets. 

In May 2005 the Spanish Parliament gave to the Government the possibility of supporting dialogue processes with ETA given that appropriate conditions to end violence occur \cite[Resoluci\'on 34, p. 13]{Congreso}, what was a change in the anti-terrorist policy because the last time the Spanish Government tried to negotiate with ETA was when the Popular Party (conservative-center wing) won elections in March 1996. Even though the LPP is still into force, the opened dialogue process led, in our opinion, to two main political facts: new left-wing nationalist parties were allowed again to present candidates in elections, and ETA began its "permanent and verifiable cease-of-fire", in Jan 2011 \cite{ceaseFire}. 

In regard to the elections, in local elections in May 2007, the left-wing nationalist party \textit{EAE-ANV} was allowed to present candidates in some villages and cities. In local elections in May 2011, without  restrictions, the coalition \textit{BILDU-EA-ALTERNATIBA} obtained around $25\%$ of votes and was the second most voted party in the Basque Country \cite{mir}. In Spanish Parliament elections in Nov 2011, under the name \textit{AMAIUR}, was also the second most voted party in the Basque Country with $24.11\%$ of votes obtaining $6$ seats \cite{mir}. Finally, in the Basque Country Parliament regional election, under the name of \textit{EH-BILDU}, they obtained $25\%$ of votes and $21$ seats \cite{ev2012}. \label{datosElec}

\section{Euskobarometro and the attitude of the Basque population towards ETA}
In order to perform a complete and thorough study, we need to know data about the sociological situation in the Basque Country respect to ETA and to do that, one of the most recognised sources is the Euskobarometro. Euskobarometro \cite{eusko} ("Basque-barometer") is a sociological statistical survey in the Basque Country. It is conducted by the Department of Political Science of the University of the Basque Country and it is based on personal interviews at home, asking questions about the sociological current issues, including ETA.  

As we mentioned in the Introduction, popular support is an important enabler for radical violent organizations and it may be crucial for their survival. At the same time, extremist groups have also an impact in the societies where they are inserted, especially if those groups are engaged in violent activities. Social and behavioral scientists try to find clues about how that interaction may affect those people, either at the group or at the individual level. 

Thus, taking data series from Euskobarometro survey, in Figures \ref{datosS}, \ref{datosR} and \ref{datosA} we can see the evolution of Basque Country population with respect to their attitudes towards ETA since May 1995 until Nov 2012 (question 20 of the Euskobarometro).

\begin{figure}[ht]
  \centering
  \includegraphics[scale=0.5]{IMG/DataGraphS.pdf}
  \caption{Percentage of Basque Country population with an attitude of support towards ETA since May 1995 until Nov 2012. Vertical lines correspond to remarkable dates.}
  \label{datosS}
\end{figure} 

\begin{figure}[ht]
  \centering
  \includegraphics[scale=0.5]{IMG/DataGraphR.pdf}
  \caption{Percentage of Basque Country population with an attitude of rejection towards ETA since May 1995 until Nov 2012. Vertical lines correspond to remarkable dates.}
  \label{datosR}
\end{figure}

\begin{figure}[ht]
  \centering
  \includegraphics[scale=0.5]{IMG/DataGraphA.pdf}
  \caption{Percentage of Basque Country population with an attitude of indifference/abstention towards ETA since May 1995 until Nov 2012. Vertical lines correspond to remarkable dates.}
  \label{datosA}
\end{figure}

The vertical lines in Figures \ref{datosS}, \ref{datosR} and \ref{datosA} correspond to remarkable dates: in Sep 1998, ETA announced an unlimited cease-fire without conditions; in Dec 1999, ETA announced the end of the cease-fire and the resume of violence; in Jun 2002 the LPP was passed; in May 2005 the Spanish Parliament gave to the Government the possibility of supporting dialogue with ETA; in May 2007 the left-wing nationalist party could present candidates again; in Jan 2011 ETA announced a permanent cease-fire. 

Observe that large jumps in the Rejection population correspond to large jumps in the Indifferent/Abstention population, in the opposite direction. Furthermore, note that, since Jun 2002 when the LPP passed, the percentage of Basque people having a supporting attitude towards ETA is around $4\%$.

It is remarkable that events with high repercussion in social media hardly produced changes on the attitude of Basque Country population towards ETA. As examples, we mention the kidnapping during 532 days (Jan 1996 - Jul 1997) of Jos\'e Antonio Ortega Lara \cite{OLara1, OLara2}, or the kidnapping and murdering of Miguel \'Angel Blanco \cite{MABlanco1, MABlanco2, MABlanco3} Jul 10th-12th, 1997, or, even though not related to ETA, the March 11th, 2004 Madrid train bombings \cite{4trama, 11m}.

\section{Current situation}\label{reduccion}
As a consequence of the LPP, at this moment, there is a singular situation where: 

\begin{itemize}
\item It seems that ETA has become weaker \cite{Harmon, debil1, debil2, debil3}.
\item Most of the Basque society ($87\%$) considers ETA a lot or quite negative \cite[p. 57]{euskoNOV2012}.    
\item Basque left-wing nationalist parties are the second most voted \cite{mir, ev2012}. We mentioned above when we described the left-wing nationalist electoral results (page \pageref{datosElec}).
\item Relevant people in the Basque Country, for the first time, dare to say that ETA may be an obstacle to get the objectives of Basque nationalism \cite{obstaculo}. 
\end{itemize}

Despite the growing of the left-wing nationalism, the supporting attitude towards ETA does not grow, therefore, not all the Basque people who vote nationalist or pro-independence options agree with ETA. In fact, in \cite[p. 57]{euskoNOV2012} the authors of the Euskobarometro report say that ETA's permanent cease-fire has been the key of the electoral recovery of the left-wing nationalist parties and currently, the supporters of ETA are only around $14\%$ of the people who vote left-wing nationalist parties \cite[p. 56-57]{euskoNOV2012}. The remainder $86\%$ does not have a support attitude towards ETA. However, these supporters and their environment, hypothetical pool of candidates willing to join the organization in upcoming years, have had a certain stability in the last decade. Therefore, it would be interesting to study and predict the dynamic evolution of the three populations (Supporters, Rejectors and Abstentionists) over the next few years in order to know if the current scenario (anti-terrorist policies, population expectations, political situation, etc.) could foster ETA eventually dying out by reducing the Supporters population.

In order to address this challenge, we should realise that there was a network of organizations related to ETA and its social wings as religious groups, training programmes, mass media (Egin), ecologist groups (Eguzki), women organizations (Egizan), anti-drug programmes (Azkagintza), students (IA and OMEV) and children (Kimuak and Champi\~n\'on) groups, Basque language programmes (AEK and Euskalherrian Euzkaraz), international solidarity (Askapena) and prisoners and refugees \cite{llera}. Also, these organizations had buildings and pubs where meet and develop their activities \cite{miguel, sentencia}, providing a closed way of life to their members based on its political ideology. Thus, ETA and its environment can keep constant pressure on the Spanish government and the remainder Basque (mainly the non-nationalist) population. Some of them, finally, cannot take the pressure and have to migrate \cite{exilio2, exilio3, exilio1}. These points give strength to approach the problem using epidemiology techniques. 

ETA appears in several publications, in the economic study of the impact of terrorism \cite{Abadie, Lbuesa, Enders1, Enders2}, in statistical techniques applied to understand its terrorist activities \cite{Barros2, Barros3, BARROS} or in interrelated issues of terrorism, human rights and law enforcement in a context of political change  \cite{Alonso}. In \cite{miguel} the author presents a functional approach to radical violent groups, focusing in ETA and its environment, and proposing a model describing the process by which young people are recruited by these groups. More details and history of ETA can be found, for instance, in \cite{Barros3, BARROS, Ldominguez, Lreinares}.
