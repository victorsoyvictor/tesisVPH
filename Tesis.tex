
\documentclass[12pt,a4paper]{report}
\usepackage[export]{adjustbox}
\usepackage{booktabs}
\usepackage{amsmath}
\usepackage{amsfonts}
\usepackage{color,eurosym}
\usepackage{amssymb}
\usepackage{float}
\usepackage{graphicx}
\usepackage[hyphens]{url}
\usepackage{hyperref}
\usepackage{enumitem}
\usepackage[table,xcdraw]{xcolor}
\hypersetup{
    colorlinks,
    citecolor=blue,
    filecolor=blue,
    linkcolor=blue,
    urlcolor=blue
}
\newtheorem{remark}{\textbf{Remark}}[section]
%Cambiar los colores de blue a black para la impresion final

%\topmargin -1cm
%\textheight 24cm
%\oddsidemargin -0.5cm
%\textwidth 17cm

\begin{document}

%PRIMERA PAGINA: La portada
\thispagestyle{empty}

\begin{figure}[h]
  	\centering
  	 \includegraphics[width=0.3\textwidth]{IMG/escudo_upv_transp.pdf}
\end{figure}
\begin{center}
\textbf{\normalsize Universitat Polit\`{e}cnica de Val\`{e}ncia}\\
\textbf{\normalsize Departament de Matem\`{a}tica Aplicada}

\vspace{1cm}

\scriptsize{\textbf{PhD. THESIS}}

\vspace{0.5cm}

\begin{center}
\textbf{\Huge Building networks of sexual partners. Application for the study of the transmission dynamics of Human Papillomavirus (HPV)}
\end{center}

\vspace{3cm}

\begin{tabular}{ccc}
\textbf{Ph.D. CANDIDATE} 				& \hspace{0.7cm} &\textbf{ADVISORS} \\
 										& \hspace{0.7cm} &\\
 										& \hspace{0.7cm} &\normalsize{Dr. Luis Acedo Rodr\'{i}guez \hfill} \\
										& \hspace{0.7cm} &\\
\normalsize{D. V\'{i}ctor S\'{a}nchez Alonso} 	& \hspace{0.7cm} & \normalsize{Dr. Rafael Jacinto Villanueva Mic\'{o} \hfill } \\ 
										& \hspace{0.7cm} &\\
 										& \hspace{0.7cm} & \normalsize{Dr. Francisco Javier Villanueva Oller \hfill} \\ 
\end{tabular} 

\vspace{2cm}

\normalsize{\textbf{Valencia - November 2018}}

\end{center}

\newpage

%SEGUNDA PAGINA: El certificado

\vspace{3cm}
Dr. Rafael Jacinto Villanueva Mic\'{o}, professor at the Universitat Polit\`{e}c\-ni\-ca de Val\`{e}ncia, Francisco Javier Villanueva Oller professor at the Universidad Rey Juan Carlos and Dr. Luis Acedo Rodr\'{i}guez professor at the Universitat Polit\`{e}c\-ni\-ca de Val\`{e}ncia.
\vspace{1.5cm}

\textbf{CERTIFY} that the present thesis entitled \textit{Application for the study of an Human Papillomavirus (HPV) dynamic transmission model in lifetime sexual partner networks} has been performed under our supervision in the Department of Applied Mathematics at the Universitat Polit\`{e}cnica de Val\`{e}ncia by V\'{i}ctor S\'{a}nchez Alonso. It constitutes her thesis dissertation to obtain the PhD degree in Mathematics.

In compliance with the current legislation, we authorize the presentation of this dissertation signing the present certificate.

\vspace{1.5cm}

\begin{center}
Valencia, \today
\end{center}

\vspace{5cm}

\begin{center}
\begin{tabular}{ccccc}
Rafael Jacinto	& \hspace{1.4cm} &  Francisco Javier & \hspace{1.4cm} & Luis \\
Villanueva Mic\'{o}	& \hspace{1.4cm} &  Villanueva Oller	& \hspace{1.4cm} & Acedo Rodr\'{i}guez
\end{tabular} 
\end{center}

\newpage

%TERCERA PAGINA: Los resumenes

\chapter*{Abstract}
Sexually Transmitted Diseases (STD) have been a major public health threat for a long time in human history. Modern concerns about STD began with the pandemic of syphilis which spread over Europe in the early sixteenth century. 

The Human Papillomavirus (HPV) is the direct cause of more than half million new cases of cervical cancer, the second most common malignancy among women and a leading cause of cancer death worldwide. It also causes anogenital warts. It is estimated that the probability of transmission of HPV is 40-50\% per contact. The networks of sexual contacts in human populations are crucial to the spread of sexually transmitted diseases (STDs).

Working in large networks applied to epidemiological-type models has led us to design a simple but effective computed distributed environment to perform a large amount of model simulations in a reasonable time in order to study the behavior of these models and to calibrate them. Finding the model parameters that best fit the available data in the designed distributed computing environment becomes a challenge and it is necessary to implement reliable algorithms for model calibration.

We are able to simulate our model and carry out vaccination campaigns in order to get conclusions concerning the best strategies. This information can be useful for Health Care, policy makers and pharmacological industries to save time and money.


\chapter*{Resum en Valenci\`a}

\newpage

%INDICES
\tableofcontents
\listoffigures
\listoftables

%El cap�tulo 1 debe ser la introducci�n al VPH que ya redactaste. 
%El cap�tulo 2 debe describir la construcci�n de la red y la din�mica de transmisi�n como est� en el art�culo de viruses. 
%El cap�tulo 3 debe ser el 2 del git de arriba donde se describe el calibrado. 
%El cap�tulo 4 el 3 del report donde se describe la red din�mica y la vacunaci�n. 
%El cap�tulo 5 de la tesis deber�a ser el caso australiano tal cual del art�culo de viruses.

%Cap�tulo 1 - ya redactado

\chapter{Introduction}\label{CAPINTRO}

The structure and properties of networks of sexual contacts in human populations is a topic of key interest in connection with the spread of sexually transmitted diseases (STD). However, this problem has received scarce attention and the modelling of STD epidemiology is usually based upon theoretical
proposals in terms of the network structure usually unvalidated. The goal of this paper is to provide a method to build a reasonable network structure from statistical data from the Health and Sexual Habits Survey in Spain.
In particular, we seek to satisfy the constraints imposed by the distribution of the number of partners for both males and females. We show that such a network can be obtained by a matching method of the bipartite graph of males and females which takes into account the preassigned degree of connectivity. 

In order to perform the pairing we apply the principle of psychological similarity by considering that people with a given tendency to acquire a certain number of partners tend to form relationships with other people with the same habits. 

This quantity is measured by a distance function \(x;y\).
The method is applied to infer the structure of networks with up to 50; 000 people, which is larger than any other one analyzed in previous field studies.

Sexually transmitted diseases have been a major public health threat for a long time in human history. Modern concerns about STD began with the pandemic of syphilis which spread over Europe in the early sixteenth century. Nowadays, syphilis still affects twelve million people all around the world every year, causing 113,000 deaths in 2010 [1]. Gonorrhea spreads at a rate of AIDS 88 million cases each year [2], while human papillomavirus is thought to be the direct cause of 561,200 new cervical cancer cases only in 2002 [3]. The global pandemic of caused by the lentivirus HIV is perhaps the most acute and widespread in human history since it has already caused 36 million deaths worldwide and it has a pool of 35.3 million people infected
by HIV in 2012.

This kind of diseases are more likely to produce large-scale pandemics than other transmissible diseases, respiratory or other, because the efficacy of sexual contacts for the infection is large and the infectious agent has long latency periods as in the case of HIV. Consequently, nor the carrier neither his/her partner is not aware of their exposure to it. For example, it has been estimated that around 40-50\% contacts are capable of transmitting HPV[5]. Moreover, some STDs are caused by oncoviruses such as Hepatitis B or HPV which increase the death rate of people who develop the disease.





Fowler \cite{CF}, for instance, it does not seem so clear in Political areas of knowledge and this will lead us to propose a specific justification in Section \ref{2.2} of this dissertation.

The understanding of the transmission dynamics of such a type of behavior increases our knowledge of the mechanisms behind the evolution of cultural norms and values. To do this, mathematical modelling is a tool that may help to predict the evolution of the extreme groups over the time, eventually disappearing or establishing themselves.

This area of study has been active since the September 11th attacks, not only in the study of the behavior of extreme groups, but also in how to handle bio-terrorist threaten \cite{libroF}.

The use of the type-epidemiological approach is not new but the few papers that can be found nowadays in the literature are mainly based on the Castillo-Ch\'avez \& Song's work \cite{Fanatismo} and the antecedent \cite{GoodIdea} where the authors propose a model of spreading ideas. Some of these works are \cite{stauffer1, stauffer2} where they present network versions of model introduced in \cite{Fanatismo} or \cite{Colomb} where the authors deal with the case of the insurgency in Colombia using predator-prey modelling techniques or in \cite{cherif} where a mathematical model of the dynamics of radicalization process on socio-spatial networks is studied. Also, there are some communications in congresses as \cite{Arney, Jairo, Jairo0}. Moreover, the introduction of non-linear models of electoral change have also been proposed \cite{WEISBERG}.

Even though some of the above papers are based in \cite{Fanatismo}, in this paper, Castillo-Ch\'avez \& Song say that "... we study the dynamics of the spread of extreme behavior as some type of epidemiology contact processes. We are aware that our approach and the associated caricature model (as most sociological models) can be easily derailed or deconstructed. We hope that our efforts are not taken that lightly, as we believe that epidemiological models still represent a reasonable starting point for the study of the spread and growth of behavior that are the engine behind most acts of terrorism".

The above comment is pertinent not only because the authors know that the model they propose is a limited first approach, but also because mathematical modelling in Social Sciences, in general, and in a complex phenomenon as terrorism or fanaticism in particular, is not an easy task. The reasons of this difficulty may lie in the uncertainty and complexity of the social phenomena as well as in the novelty of using dynamical models based on differential equations in an area where Statistics is practically the only methodological tool for quantification. This explains the effort and, sometimes, incomprehension when mathematicians try to collaborate with professionals in Social Sciences trying to find a common language to understand each other.

We should also say that all the papers referred above propose theoretical models where a dynamic analysis is performed. The dynamical analysis is a powerful tool, nevertheless, as Castillo-Ch\'avez \& Song wrote in \cite{Fanatismo} "... even though the core population is on its way of extinction, it can still experience grow and expand in finite time before it begins to decay", and this \textit{finite time} may be long. This fact leads us to use real data to build models to work with, providing an additional value to our work because we will be able to describe the groups behavior not only in the long run, but also in medium and short term, what may be more realistic if the objective is to use the models to simulate new policies related to model parameters and see the effects. Furthermore, working with real data constitutes an additional effort controlling the uncertainty in parameter estimation and also the propagation of the mentioned uncertainty in the model predictions.   

In this dissertation, we focus our study in the especial situation occurring in the Basque Country \cite{PaisVasco}, a Northern Spanish region, where the Basque revolutionary organisation ETA (Basque for "Basque Homeland and Freedom") \cite{wETA} has been using different forms of terrorism to achieve its political goals during the last 50 years. The existence of an extreme left wing terrorist organization \cite{PC} in a democracy as is Spain is unique all around the world, except, maybe in Colombia, where the jungle and mountains help the FARC to establish, in some hidden areas, its own government \cite{Lmiguel}.

ETA declared a cease-fire in Jan 2011, at this moment it is not attacking and, as we can see in Figure \ref{preocupa}, there is a reduction in the percentage of Basque people concerned about violence and terrorism.

\begin{figure}[ht]
  \centering
  \includegraphics[scale=0.7]{IMG/preocupacion.pdf}
  \caption{Percentage of Basque people concerned about violence and terrorism over the time since May 2000 until May 2013 \cite{eusko}.}
  \label{preocupa}
\end{figure}  

Despite this reduction, nowadays, around $50\%$ of Basque people do not feel free to talk about politics and around $25\%$ of Basque people have a lot or quite fear to participate actively in politics in the Basque Country (see Figure \ref{feeling}). It does not seem a typical scenario of democratic normality. In fact, there are groups related to ETA that pressure the society to achieve their goals. Moreover, we should not forget that ETA still has not abandoned violence definitively and it would not be the first time ETA declares cease-fire that breaks unilaterally and resumes the violent acts. All these facts disclose that the problem of the violence in the Basque Country, and by extension in Spain, is as up-to-date as ever and account for developing the present study.

\begin{figure}[ht]
  \centering
  \includegraphics[scale=0.7]{IMG/feeling.pdf}
  \caption{(Up) Percentage of Basque people depending on their freedom feeling to talk about politics. (Down) Percentage of Basque people depending on their fear feeling to participate actively in politics. \cite{eusko}.}
  \label{feeling}
\end{figure}    

\section{Uncertainty}
The treatment of the uncertainty in the models is going to be one of the keys in this dissertation.

Uncertainty quantification in dynamic continuous models is an emerging area \cite{maitre}. Because of the numerous complex factors that usually involve social behavior, it is particularly appropriate the consideration of randomness in this kind of models. In practice, the introduction of randomness in continuous models can be done using different approaches. Stochastic differential  equations of It\^{o}-type consider uncertainty through a stochastic process called white noise, i.e., the derivative of a Wiener process. As a consequence, this approach limitates the introduction of uncertainty to a gaussian process whose sample trajectories are somewhat irregular since they are nowhere differentiable. A more convenient approach in social modelling is to permit that input parameters can become random variables and/or stochastic processes and, therefore can follow other type of probability distributions apart from gaussian. This approach leads to continuous models usually referred to as random differential equations (r.d.e.'s). In dealing with r.d.e.'s, generalized Polynomial Chaos (gPC) is likely one of the most fruitful methods \cite{Spanos, Xiu}.

Most of the existing methods and techniques, start with the assumption that the model parameters follow a known standard probability distribution. In general, setting the probability distribution of the model parameters, standard or empirical, is a crucial and difficult task currently under study which is required for model uncertainty approaches.

Also, the computation is an important issue in dealing with uncertainty. For instance, gPC technique may not be affordable when the number of model parameters with uncertainty increases, or the interval where the mean and the standard deviation are valid may be very short \cite{Benito}. It may turn these techniques inappropriate for modelling real problems. 

On the other hand, if we consider that no information is available for setting the model parameters probabilistic distribution, techniques as bootstapping \cite{almu, NOS} or bayesian \cite{bay} are other useful approaches. 

\section{Overview of the dissertation}
Popular support is an important enabler for radical violent organizations and it may be crucial for their survival. At the same time, extremist groups have also an impact in the societies where they are inserted, especially if those groups are engaged in violent activities. Social and behavioral scientists try to find clues about how that interaction may affect those people, either at the group or at the individual level, in order to foresee subsequent dynamics. 

In this dissertation, our objective is to shed light on the dynamics about the attitude the Basque Country population have towards ETA, its goals and the means it uses to achieve them. To do that we use mathematical models and introduce uncertainty into these models step by step, in a natural way as an intrinsic part of modelling in Social Sciences. 

In particular, we focus on the events that affect the attitude towards ETA, the effect of the Law of Political Parties (LPP), the influence of the truces, the relation between ETA's supporters and the source of ETA's activists, the prediction of the evolution of ETA's supporters, etc. All the above facts involve uncertainty in the model and through the present dissertation we will explain how we introduce the uncertainty treatment in each one of the presented models, rough first and more sophisticated at the end.  

In the following, we present a content description of this PhD dissertation. Also, we report its historical developing because it explains some decisions we made during the progress of this work.
 
We start in Chapter \ref{ETA}, where we describe the organization ETA and its main facts in order to justify the interest of studying the ideological evolution of the most affected people, the Basques and the Spanish. Moreover, we introduce the Euskobarometro, a sociological statistical survey in the Basque Country that will provide us source data. Also, we will give some interesting references to understand the "Basque problem" (using ETA's language) and the role of ETA. 

Our first modeling approach is presented in Chapter \ref{paper1}. Here we classify the population depending on the party they vote, then we grouped the political parties respect to their attitude on the "independence from Spain", one of the ETA's most important goals, using the parties' electoral manifestos. Thus, using electoral data, we are able to divide the population in the Basque Country into people that:

\begin{itemize}
\item agree with ETA in the objective of independence and the use of violence to achieve it,
\item agree with ETA only in the objective of independence, without the use of violence,
\item completely disagree with ETA.
\end{itemize}

Then we present a type-epidemiological model to study the dynamics of these groups over the time and introduce uncertainty in the prediction over the next few years using a technique called Latin Hypercube Sampling (LHS). 

Part of the results presented in this chapter were published \cite{NOS1}. However, we should say that other paper related to Chapter \ref{paper1} was prepared introducing the uncertainty using LHS not only in prediction but also in model simulations of the effect of some policies over the next few years. It was sent to journal \textit{Terrorism and Political Violence} and a negative answer was received. In fact, the referees pointed out some important drawbacks as  

\begin{enumerate}
\item "Terrorism and the ideology behind terrorism are subjects that are defined by a myriad of complex ideological, social, economic and political factors",
\item "In fact, the methodology is extremely confusing, not only because of its quantitative nature, but also as a result of the questionable model employed which presents ideology as a socially transmitted epidemic disease",
\item "The author builds up his/her model on a set of sources which also reveal a limited grasp of the most relevant literature on the subject. The author completely ignores the work of key writers on the Basque conflict",
\item "There is not a single mention to other primary sources that are much more relevant than the ones used by the author. For example, the periodical sociological and political surveys produced by the Euskobarometro or the regular reports on the
violent activity of ETA produced by the Interior Ministry are key sources for anybody who wants to analyze precisely what the author wants to analyze",
\item "The author completely ignores the variety of identities within Basque society and the social support for each of these ideological stances. Instead the author wrongly simplifies such a complexity of political identities by summarizing them in the following categories:  E, non-nationalist people against independence and the use of the violence; N, nationalist people agree with independence but disagree with the use of the violence; V, nationalist people agree with independence and the use of the
violence, and A, people who do not share the above mentioned ideological characteristics or people who abstain. As regular surveys have shown for many decades, the majority of the nationalist population of the Basque Country do not advocate independence",
\item "The data on the different ideological stances among Basque population has been extracted from incomplete sources leaving out other sources which are much more up to date and relevant",
\item "In a nutshell, the article presents a very simplistic, sketchy picture of a complex, multifaceted issue and fails to provide a clear contribution to the understanding of the subject",
\end{enumerate}

and so on. Nowadays, we admit most of them, except the 2nd (of course), but the most important for us was to figure out how far was our language, method, argumentation and mind from those experts in political sciences, terrorism and extreme ideologies. 

Therefore, we needed an expert partner and we were so lucky that he found us before we started searching. The expert addressed us to study a relevant problem in the area, what was a change in our way, and the results are presented in Chapter \ref{paper2}.

About the use of LHS technique to deal with the model uncertainty, we should say that it is satisfactory as a first attempt, however we had to assume the fact that the model parameters follow a uniform distribution (because we do not have information about them) with an unknown variation we had to establish in $20\%$.  

Following the suggestions of the expert, in Chapter \ref{paper2} we study whether the "Law of Political Parties" (LPP) had an effect on attitude of the Basque population towards ETA and we tried to quantify this effect. In June 2002, the Spanish Government passed the LPP with the aim, among others, to prevent parties giving political support to terrorist organizations. This law affected the Basque nationalist party "Batasuna", due to its proved relation with ETA. Then, taking data from the Euskobarometro (Basque Country survey) related to the attitude of the Basque population towards ETA, we propose a dynamic model for the pre-LPP scenario. This model will be extrapolated into the future in order to predict what would have happened to the attitude of the Basque population if the law had not been passed. These model predictions will be compared to post-LPP data from the Euskobarometro using a bootstrapping approach in order to quantify the effect of the LPP on the attitude of Basque Country population towards ETA. 

In this chapter, the uncertainty is studied applying a bootstrapping technique to the dynamic model. Bootstrapping technique was a satisfactory technique (we give an answer to the problem), but during its application we realised that, to be applied, it has to fulfill restrictive hypotheses and it does not consider the uncertainty in the initial condition.

In the current times, where there are important political parties, mass media and organizations supporting the negotiation with ETA as the best choice to finish its activities, the Chapter \ref{paper2} supports that the opposite idea may be possible, this is, the appropriate use of the laws is an useful tool to fight against these extreme organizations and change the citizen's mind against the social pressure. Moreover, even though it is not reflected in the chapter but can be seen in Figures \ref{datosS}, \ref{datosR} and \ref{datosA}, although relevant changes and events have occurred, the support attitude towards ETA of the Basque population has hardly varied (moving around $4\%$) since LPP. That is, despite of what uses to happen with other laws which effect disappears in two or three years returning to pre-law figures with independence of the time they are into force \cite{tabaco}, the Law of Political Parties seems to have produced a permanent effect on the attitude of the Basque population towards ETA. The results presented in Chapter \ref{paper2} have been published in \cite{NOS}.

We also should say that this is one of the few works that, using dynamic models, tries to quantify the effect of a law. The only reference we know using a similar technique is \cite{tabaco} where the effect of Spanish tobacco law in 2006 is analysed.

The objective in Chapter \ref{paper3} is to analyse the evolution dynamics of the populations in Chapter \ref{paper2} using the same model from May 2005 to Nov 2012 in order to predict the future evolution of the attitude of the Basque population towards ETA, taking into account that Supporters may be considered as the main source of ETA members. To get this objective, we propose and apply a new computational technique to deal with uncertainty in dynamic continuous models without the drawbacks of LHS and bootstrapping. Considering data from surveys, the method consists of determining the probability distribution of the survey output and this allows to sample data and fit the model to the sampled data using a goodness-of-fit criterion based on the $\chi^2$-test. Taking the fitted parameters non-rejected by the $\chi^2$-test, substituting them into the model and computing their outputs, we build $95\%$ confidence intervals in each time instant capturing uncertainty of the survey data (probabilistic estimation). Using the same set of obtained model parameters, we also provide a prediction over the next few years with $95\%$ confidence intervals (probabilistic prediction).

Finally, in Chapter \ref{conclusion} we enumerate the main goals this dissertation achieved. 

%Cap�tulo 2 - de revista Viruses
\input{ConstruccionYDinamica.tex}
%Cap�tulo 3 - Validacion caso AUS de revista Viruses
%Aqui hay que mezclar sabiamente y en orden el paper de Viruses y los resultados del informe
\chapter{Australian case}\label{Australiano}

In this chapter we are going to check if we can obtain similar results to those in \cite{ali2013genital,fairley2009rapid} using the parameters of the new calibration.

As we introduced in Chapter \ref{intro} there is a decrease on the number of infected persons and the number of persons with GW is already reported for Australia after two years of administering vaccinations to young girls. These results were more impressive than predicted by continuous models.

To check the reliability of the model, we simulated the HPV vaccination campaign carried out in Australia \cite{ali2013genital}, and compared them with the actual impact published \cite{ali2013genital}. In 2007, Australian health authorities started a vaccination program for 12--13 year-old girls with a~coverage of $73\%$ ($83\%$ in the first dose, $80\%$ in the second dose and $73\%$ in the third dose). In addition, from 2007 to 2009, there was a catch-up vaccination program for women aged 13--26 with a decreasing coverage with age until $52\%$ in women aged 20--26. Their results can be summarized as follows \cite{ali2013genital}:

\begin{itemize}
	\item Two years after the vaccine was introduced, the proportion of genital warts diagnosed declined by a $59\%$ in vaccine eligible young women aged 12--26 years in $2007$, and by $39\%$ in men of the same age.
	\item No significant decline was observed in women or men older than $26$ years old, non-resident young women, or men who have sex with men.
\end{itemize}

Two different scenarios were considered to be simulated:

\begin{itemize}
	\item Scenario 1: vaccination of $83\%$ of the $14$ year-old girls (or younger girls) plus a catch-up with coverage $73\%$ for 14--26 year-old women.
	\item Scenario 2: vaccination of $73\%$ of $14$ year-old girls (or younger girls) plus a catch-up with a~vaccination coverage of $52\%$ for 14--26 year-old women.
\end{itemize}

These simulations represented the upper and lower bounds of the scenario implemented in Australia. 

\section{How to measure the decline}\label{sec:decline}%esto lo borré de otro capítulo y lo pongo ya aqui

We call $I$ the number of infected women of LR HPV 6 and/or 11 just before the starting of the vaccination campaign; we call $V = ( v_1, \ldots, v_N)$ to the number of infected women of LR HPV 6 and/or 11 every month from the starting of the vaccination program until the end of the simulation. Then,~the~vector 

\begin{equation}
100 \times \left( 1-\displaystyle\frac{v_1}{I}, \ldots, 1-\displaystyle\frac{v_N}{I} \right) \; 
\end{equation}
is a measure of the percentage of decline of number of infected women of LR HPV 6 and/or 11 after the beginning of the vaccination campaign. This will also be applied to men and MSM.

In order to compare GW data given in \cite{ali2013genital} with our model, results referred to infected women of LR HPV 6 and/or 11, we should take into account that, whether a fixed proportion of HPV 6 and/or 11 infected individuals develops warts, the percentage of decline in warts and in infected women of LR HPV 6 and/or 11 will be comparable. 

Another important issue for the natural history of the disease is the persistence of the infection \cite{campos2014updated}. Our model does not consider the persistence ``a priori'', but we derive the cases of genital warts from the number of cases of infected individuals by taking this data into account.



\section{Does our model return similar values to those in \cite{ali2013genital}?}
In this section, we are going to show figures about prevalence or decline of the percentage of women, men and MSM infected of LR 6/11, the HPV type responsible of $90\%$ of genital warts. Taking into account that genital warts, in average, use to appear 6 months after the infection, the figures about prevalence or decline will be a good estimation of the prevalence and the  decline of genital warts.  

Figure \ref{fig:prev_AUS_6_11} shows the percentage of women, men and MSM aged 14-26 infected of LR 6/11 after starting the vaccination program in both simulated scenarios. We can see the fast decrease for women and men in both scenarios from the very beginning. MSM, remain constant. 

\begin{figure}[!]
	\centering
	\begin{tabular}{cc}
		\includegraphics[width=0.5\linewidth]{IMGs/3.-Australia/Retr_muj_14_26_verr_Australia.pdf}	& 
		\includegraphics[width=0.5\linewidth]{IMGs/3.-Australia/Retr_hom_14_26_verr_Australia.pdf}  \\ 
		(a)	& (b) \\ 
		\multicolumn{2}{c}{ \includegraphics[width=0.5\linewidth]{IMGs/3.-Australia/Retr_MSM_14_26_verr_Australia.pdf} } \\ 
		\multicolumn{2}{c}{(c)} \\ 
	\end{tabular} 
	\caption{Percentage of women (a), men (b) and MSM (c) aged 14-26 infected of LR HPV 6 and/or 11 after the implementation of the vaccination program. The red lines correspond to the average and $95\%$ confidence interval for Scenario 1 and the black lines to Scenario 2.  We can see the fast decrease for women and men in both scenarios from the very beginning. However, there is not effect on MSM.}
	\label{fig:prev_AUS_6_11}	
\end{figure}

In Figure \ref{fig:decline_AUS_6_11}, we have plotted the same data as in Figure \ref{fig:prev_AUS_6_11} but from another point of view: the average percentage of decline of women and men infected of LR HPV 6 and/or 11. As the vaccination program progresses over time, the percentage of decline obviously grows. After 2 years, the model shows a decline of

\begin{itemize}
	\item Scenario 1: $72.0\%$ with CI $95\%$ $[67.7\%, 76.5\%]$ for women and $38.9\%$ with CI $95\%$ $[32.0\%, 45.5\%]$ for men. 
	\item Scenario 2: $54.8\%$ with CI $95\%$ $[48.5\%, 59.0\%]$ for women and $27.7\%$ with CI $95\%$ $[21.3\%, 34.5\%]$ for men. 
\end{itemize}

Australian reported levels of decline ($59\%$ in women and $39\%$ in men aged 14-26) will be reached by the model after

\begin{itemize}
	\item Scenario 1: $1.66$ years with CI $95\%$ $[1.5, 1.75]$ for women and $2.0$ years with CI $95\%$ $[1.75, 2.16]$ for men,
	\item Scenario 2: $2.1$ years with CI $95\%$ $[2.0, 2.33]$ for women and $2.42$ years with CI $95\%$ $[2.08, 2.83]$ for men.
\end{itemize}

\begin{figure}[!]
	\centering
	\begin{tabular}{cc}
		\includegraphics[width=0.5\linewidth]{IMGs/3.-Australia/Decl_muj_14_26_verr_Australia.pdf}	& 
		\includegraphics[width=0.5\linewidth]{IMGs/3.-Australia/Decl_hom_14_26_verr_Australia.pdf}  \\ 
		(a)	& (b) 
	\end{tabular} 
	\caption{Percentage of decline of women (a) and men (b) aged 14-26 infected of LR HPV 6 and/or 11 (and consequently of genital warts) after the implementation of the vaccination program. The red lines correspond to the average and $95\%$ confidence interval for Scenario 1 and the black lines to Scenario 2. After 2 years, the model shows a decline of $72\%$ for Scenario 1 and $54.8\%$ for Scenario 2, in average, for women and $38.9\%$ for Scenario 1 and $27.7\%$ for Scenario 2, in average, for men.}
	\label{fig:decline_AUS_6_11}
\end{figure}

No significant impact on the rate of infection was observed in men aged 27-64 and women 2 years after the implementation of the vaccination program (Figure \ref{fig:decline_AUS_6_11_27_64}) and the same in women agreeing the observations reported in \cite{ali2013genital}. It can be explained by the fact that, usually, individuals have sexual intercourses with people more or less the same age.

Then, our model predict figures close to the ones given in \cite{ali2013genital}.

\begin{figure}[!]
	\centering
	\begin{tabular}{cc}
		\includegraphics[width=0.5\linewidth]{IMGs/3.-Australia/Decl_muj_27_64_verr_Australia.pdf}	& 
		\includegraphics[width=0.5\linewidth]{IMGs/3.-Australia/Decl_hom_27_64_verr_Australia.pdf}  \\ 
		(a)	& (b) 
	\end{tabular} 
	\caption{Percentage of decline of women (a) and men (b) aged 27-64 infected of LR HPV 6 and/or 11 (and consequently of genital warts) after the implementation of the vaccination program in both scenarios. The red lines correspond to the average and $95\%$ confidence interval for Scenario 1 and the black lines to Scenario 2. Notice that, in average, no significant decline appears in the 5 years after the implementation of the vaccination program.}
	\label{fig:decline_AUS_6_11_27_64}
\end{figure}

\section{Study of the herd immunity effect over HPV LR infection}
The herd immunity effect in both scenarios is shown in Figure \ref{fig:decline_AUS_6_11_14_64} for women, men and MSM. Notice that, in men and MSM, any decline is due to herd immunity. The decline in the whole female population appears when the lines representing their decline are over the vaccination lines (green for Scenario 1 and orange for Scenario2) also shown in this figure. We see that the herd immunity effect starts after 

\begin{itemize}
	\item for women
	\begin{itemize}
		\item Scenario 1: $0.58$ years with CI95\% $[0.0, 22.1]$.
		\item Scenario 2: $0.58$ years with CI95\% $[0.0, 22.1]$.
	\end{itemize}
	\item for men
	\begin{itemize}
		\item Scenario 1: $0.0$ years with CI95\% $[0.0,0.83]$.
		\item Scenario 2: $0.0$ years with CI95\% $[0.0,0.83]$.	
	\end{itemize}
\end{itemize}

The herd immunity effect starts very quickly for men. For MSM, there is not a clear herd immunity effect because the decline is stable over the time. For women, we can see that, practically, the CI$95\%$ decline lines are over the vaccination lines in both scenarios. This means that, in the worst case, only the vaccinated women will be protected and in the best case, almost all women will be protected by vaccination or by herd immunity effect.

\begin{figure}[!]
	\centering
	\begin{tabular}{cc}
		\includegraphics[width=0.5\linewidth]{IMGs/3.-Australia/Decl_muj_14_64_verr_Australia.pdf}	& 
		\includegraphics[width=0.5\linewidth]{IMGs/3.-Australia/Decl_hom_14_64_verr_Australia.pdf}  \\ 
		(a)	& (b) \\ 
		\multicolumn{2}{c}{ \includegraphics[width=0.5\linewidth]{IMGs/3.-Australia/Decl_MSM_14_64_verr_Australia.pdf} } \\ 
		\multicolumn{2}{c}{(c)} \\ 
	\end{tabular} 
	\caption{Percentage of decline of women (a), men (b) and MSM (c) aged 14-64 for the vaccination program in Australia. The red lines correspond to the average and $95\%$ confidence interval for Scenario 1 and the black lines to Scenario 2. In the figure (a), green and orange lines correspond to women vaccination percentage for Scenario 1 and 2, respectively. Notice that the herd immunity effect contributes to the decline in the number of infections in men and the decline in the number of infections for unvaccinated women. This latter can be seen when the decline lines are over the vaccination line. However, any herd immunity effect can be seen in MSM.}
	\label{fig:decline_AUS_6_11_14_64}
\end{figure}

Notice that the herd immunity effect is very clear within the CI$95\%$ both for women and men, but it does not appear in the MSM population. In the best case scenario, the MSM subpopulation achieves a constant protection level of $10\%-15\%$. This could be attributed to the way in which the MSM individuals are connected: with a very large number of LSPs among them and some casual links with women with large LSPs.

\section{Discussion}
The random network of sexually transmitted HPV including up to $100,000$ nodes, was developed to fit the data of surveys concerning the number of sexual partners throughout life. Standard continuous models are insufficient to accurately predict transmission because they do not account for the individual to individual transmission of the infection, the role of hubs in disseminating the virus through the rest of the population and neither the vaccination campaigns targeting specific groups of individuals.

This network has successfully been applied to the stable state of infections by LR and HR HPV genotypes in Spain  \cite{Acedo2017}. In this study we mimicked the results found in the HPV vaccination campaign in Australia \cite{ali2013genital}, and showed very reliable results. 

Models based upon continuous differential equations predict a slower decrease in the number of infected individuals after implementing similar vaccination campaigns \cite{elbasha2007model}. Hence, the case of the HPV vaccination in Australia provides one of the best real scenarios for testing new network models in mathematical epidemiology. There is an on-going debate on the pertinence of an approach based upon networks on epidemiology \cite{Eubank} and this work contributes to show the necessity of such an approach in many cases, in particular, in those corresponding to STD.

To validate the model, we used the Australian experience, with two different vaccination coverages: routinely vaccination campaign for $12-13$ year-old girls with a coverage of $73\%$ and $83\%$ and a catch-up program in the $14-26$ age group with an average coverage of $52\%$ and $73\%$. This program revealed an important herd effect \cite{ali2013genital}, so that vaccination decreased the incidence of genital warts (GW) even in the non-vaccinated men because of the protection of infection conferred by the vaccine, and the decreased transmission of the virus.

The model predicted a fast decline in the number of infections parallel to the decline in the number of GW in Australia with very similar values. However, this model was built with Spanish data on sexual behavior \cite{INE} and prevalence of HPV infection \cite{castellsague2012prevalence}, that might differ to the Australian one, and may explain the minor differences found between the model and the actual data published. Herd immunity in this model of STD is predicted much sooner than in other highly transmitted aerial transported infectious diseases as influenza or RSV, due to the structure of the network. This supports the need to build appropriate LSP networks. 

Other models have also predicted the protection of males by vaccinating girls and women, but only for men, as the model used by Bogaards et al. \cite{bogaards2015direct}. This model uses Bayesian techniques to study the herd immunity effect. However, in contrast with our model, it does not take into account the dynamics of the HPV transmission, the importance of age-groups and the different roles they play in the propagation of these viruses or the links among the MSM subpopulation and the heterosexual network. In this sense, a network model is required to study the impact of the vaccination strategies in short, medium and long time scales.

Vaccination strategies should seek an optimal effectiveness and efficiency. In this case, it can be seen the quick apparition of the herd immunity effect on males and females only vaccinating women. However, the herd immunity effect does not appear in  MSM. This can be the consequence of the large LSP numbers for MSM and their casual connections with women with large LSP numbers in the heterosexual subnetwork.

The model considers a quiet close community, where there is not much contact with other communities. This may not be the case in Spain which in $2016$ received over $75$ million tourists \cite{INEturismo}, representing almost the double of the number of Spanish inhabitants, and when sexual contacts are frequent. This may bias the results, as the herd immunity in Spain may not be so clear as in countries with less tourism. Nevertheless we will study the effect of tourism in Spain.

Another issue that we must take into account, is the modelling of the population with a high number of contacts because these individuals are hubs in the network whose vaccination may induce a faster decline of the virus prevalence. Our approach is rather conservative in the assignment of LSP for men and women with $10$ or more links because we assume that all of them have similar LSP. However, it is expected that individuals with extreme values of LSP are favouring the transmission of HPV in such a way that  a targeted vaccination can show its benefit in a very short time.

%Cap�tulo 3 - calibrado del articulo Report HPV RJVM August 25, 2018
\input{Calibration.tex}
%Cap�tulo 4 - dinamica y vacunacion del articulo Report HPV RJVM August 25, 2018
\chapter{Making Dynamic Sexual Contacts Network and adding Vaccination Campaigns}\label{DinamicaYVacunacion}
\section{Including new features into the model}
\subsection{Introducing age and LSP dynamics into the LSP model}
The computational model presented so far is static, that is, once we have assigned the LSPs, they do not change over the time and the nodes do not increase their age. We modulate the possibility of a contagion introducing global probabilities that determine if the existence of a LSP implies sexual intercourses in a given time step per age group $14-17$, $18-29$, $30-39$ and $40-65$. Also, note that the available data about LSPs come from $2003$ and the data of prevalence appears in a study (CLEOPATRA \cite{castellsague2012prevalence}) conducted in $2007-2008$. The sexual behavior has changed in the last $10-15$ years and we would like to include it in our model in some way. Nevertheless, newer data about LSPs and prevalence in Spain are not available.

Thus, in order to introduce age and LSP assignation dynamics in a simple way, we are going to \textit{move} the available data over the time. Figure \ref{fig:dinam1} shows the constant age distribution that do not varies over the time.

\begin{figure}[h!]
	\centering
	\includegraphics[width=0.7\linewidth]{IMGs/2.-New_features/Dinam_1.pdf}
	\caption{Static LSP network. The ages remain constant over the time. The time only affects the HPV contagion/clearing dynamics.}
	\label{fig:dinam1}
\end{figure}

Then, we evolve the ages of the nodes over the time. As long as the nodes do not change the age group, there is nothing to do. However, how do we \textit{recycle} the nodes that turn 65? And how do we transform the nodes in the first age group $14-29$ when they turn 30?  

To answer the first question, we propose to

\begin{itemize}
	\item preserve its sex in order to maintain the proportion males/females, 
	\item assign $14$ years old to the node, erase all its LSPs and transform it in susceptible,
	\item assign new LSPs according to the LSPs of the age group $14-29$ following the assortativity property.
\end{itemize}

To answer the second question, we propose to

\begin{itemize}
	\item preserve its sex and the infectious state, 
	\item erase all its LSPs and assign new LSPs according to the LSPs of the age group $30-39$ following the assortativity property.
\end{itemize}

This way, as time goes on, we will have the LSP structure by ages as we can see in the Figure \ref{fig:dinam2}.

\begin{figure}[h!]
	\centering
	\includegraphics[width=0.7\linewidth]{IMGs/2.-New_features/Dinam_2.pdf}
	\caption{Dynamic LSP network. The ages evolve over the time. The age group 40-65 and their sexual behavior disappear when people grow.}
	\label{fig:dinam2}
\end{figure}

If the node is MSM (male who have sex with males), we perform the above procedure taking into account the division of age groups given in \cite{Durex2002}, $14-19$, $20-24$, $25-29$, $30-39$, $40-49$, $50-59$ and $60-64+$.

Taking into account that the global number of LSP in the age groups $40-65$ is less that in the age group $30-39$, as the time goes on, the global number of LSP in the whole network will increase and the transmission ot HPV will also increase. However, the proposed approach may be considered as very conservative because the sexual behavior change in the last $10-15$ years seems to result in an increase of the sexual intercourses greater than the number that we can approach with the proposed dynamic network. Nevertheless, the lack of data do not allow us to quantify the mentioned change.

%, but we take advantage of the static model building and the calibrated model parameters. Furthermore, as we mentioned before, we do not have recent data about sexual behavior in Spain.

\subsection{When should we start the vaccination schedule?} 
When a realization runs, we use $500$ months to stabilize the static network, then, we change to a dynamic network. Thus, a key point arises and it is to decide in which time instant starts the vaccination schedule. We have performed a simulation with the selected $30$ sets of parameters with $2300$ months (around $191.6$ years) where the network turns dynamic from the month $500$. In the Figure \ref{fig:Estudio_ciclos} we can see the levels of prevalence predicted by the model for 18-64 years old men and women for HR and LR, over the next years.

\begin{figure}[h!]
	\centering
	\includegraphics[width=\linewidth]{IMGs/2.-New_features/Estudio_ciclos.pdf}
	\caption{Mean and $95\%$ confidence intervals of the prevalence for 18-64 men and women for HR and LR. Vertical dashed lines indicate milestones of interest. Note the oscillations of the prevalence levels. The blue and green vertical lines correspond to the peaks and valleys of the oscillations. The magenta line, points the month 500 when dynamic network starts.}
	\label{fig:Estudio_ciclos}
\end{figure}

The Figure \ref{fig:Estudio_ciclos} shows the mean and $95\%$ confidence intervals of the $30$ simulations for the prevalence of men and women for HR and LR. The horizontal axis indicates the month and the vertical axis the percentage of infected. As it can be seen, there are oscillation in the evolution of the prevalence.

The vertical lines correspond to:
\begin{itemize}
	\item magenta: month $500$, the static network turns dynamic;
	\item blue: months $980$, $1580$ and $2180$, and point out the peaks in the oscillations in the means and the percentiles. Between them, there are $600$ months ($50$ years), the time of a complete generation in the model;
	\item green: months $1280$  and $1880$, and point out the valleys. Between the valleys there are also $50$ years, and $25$ years between a peak and a valley. 	
\end{itemize}

In some test runs with vaccination, we have observed that, if we start the vaccination schedule when the prevalence is in the decreasing part of the oscillation, the herd immunity appears very much sooner that when we start the vaccination schedule in the increasing part.

If we take into account that in the paper \cite{Ali}, the authors reports extraordinary results in Australia, where two years after the vaccine was introduced, the proportion of genital warts diagnosed declined by a $59\%$ in vaccine eligible young women aged 12--26 years in $2007$, and by $39\%$ in men of the same age, we conjecture that, if there are oscillations in the prevalence of HPV over the time, they have have taken advantage of a decreasing part of the oscillation.

Recall that our goal is to determine the appropriate month where start the vaccination schedule, trying to save computations and favoring the apparition of the herd immunity effect as soon as possible, in order to minimize the opposite effect when the oscillation is in the increasing trend if we have vaccinated a large enough number of individuals.

The oscillations in all the cases, men, women, HR and LR, are very similar, as we can see in the Figure \ref{fig:Estudio_ciclos}, except, maybe in some upper percentiles. Also, there are similarities from the first peak in valleys and peaks. Therefore, in order to take the maximum advantage of the decreasing trends and to save computation, we are going to select the earliest peak, the month $980$, as the starting point for the vaccination.

\subsection{Introducing vaccination} 
For our simulations, we are going to consider the vaccine GARDASIL9 \cite{gardasil9}, that protects against the HPV types 6/11/16/18/31/33/45/52/58. The two first (6/11) are LR and the remainder are HR. Thus, GARDASIL9 protects against $90\%$ of genital wart cases and $90\%$ of the cancer cases \cite{Hartwig2015}. Although GARDASIL9 may protect partially against other HPV types, for modelling purposes we assume that it does not happen. Therefore, it would be interesting to introduce changes in order to monitor if a node is infected by HPV types included in GARDASIL9 or not, and then, to simulate accurately the protection effect of the vaccine.

Following the study conducted in \cite{castellsague2012prevalence}, if a woman is infected by HPV LR, the probability to be only infected by 6/11 is $34.23\%$, $63.06\%$ only infected by others than 6/11 and $2.70\%$ to be infected by both. Also, if a woman is infected by HPV HR, the probability to be infected only by 16/18/31/33/45/52/58 is $30.44\%$, $23.66\%$ only infected by others than 16/18/31/33/45/52/58 and $45.90\%$ by both. Due to the lack of information about men, we will also use the above percentages for men.

Then, before starting the vaccination, we label men and women as infected of HPV LR 6/11 or infected of HPV LR other than 6/11 or infected of both, following the above percentages. Analogously, we label men and women as infected of HPV HR 16/18/31/33/45/52/58 or infected of HPV HR other than 16/18/31/33/45/52/58 or infected of both.

Once these assignments are done, we continue with the HPV transmission dynamics including the vaccination, taking into account the new labels we included. If a node has been vaccinated, it can be infected by the types of HPV different to those that GARSDASIL9 protects and, in this case, they will never be infected of 6/11 nor 16/18/31/33/45/52/58. If a node has not been vaccinated, can be infected by any HPV type. 

The assumed effectiveness of the vaccine is $96.5\%$.

\subsection{Introducing vaccination loss protection}
In the previous section we assume that the protection of GARDASIL9 is forever. In fact, until now, people vaccinated by GARDASIL (previous version of GADASIL9) do not have experienced any loss in the protection. But this does not mean that it could happen in the future. 

In fact, we want to simulate the worst possible scenario, that is, the sudden drop to zero of the protection. To simulate this possibility, we will introduce a new parameter that represents the time after the vaccination where the protection is complete. Therefore, after this time, the vaccinated individual will behave as a non-vaccinated individual.

\subsection{Introducing variations in the vaccination coverage}
One of our goals is to simulate scenarios where variations in the vaccination coverage have been occurred and we want to study the effect in the global protection against HPV due to these variation of the coverage. To simulate these scenarios, we will include into the model vectors of coverage, indicating the vaccine coverage every month, and vaccinating the people following these variable coverages. 

\chapter{Conclusions}
The random network of sexually transmitted HPV including up to $100,000$ nodes, was developed to fit the data of surveys concerning the number of sexual partners throughout life \cite{Acedo2017,DezDomingo2017}. Standard continuous models are insufficient to accurately predict transmission because they do not account for the individual to individual transmission of the infection, the role of hubs in disseminating the virus through the rest of the population and neither the vaccination campaigns targeting specific groups of individuals.

This network has successfully been applied to the stable state of infections by LR and HR HPV genotypes in Spain  \cite{Acedo2017}. In this study we mimicked the results found in the HPV vaccination campaign in Australia \cite{ali2013genital}, and showed very reliable results. 

Models based upon continuous differential equations predict a slower decrease in the number of infected individuals after implementing similar vaccination campaigns \cite{elbasha2007model}. Hence, the case of the HPV vaccination in Australia provides one of the best real scenarios for testing new network models in mathematical epidemiology. There is an on-going debate on the pertinence of an approach based upon networks on epidemiology \cite{Eubank} and this work contributes to show the necessity of such an approach in many cases, in particular, in those corresponding to STI.

To validate the model, we used the Australian experience, with two different vaccination coverages: routinely vaccination campaign for $12-13$ year-old girls with a coverage of $73\%$ and $83\%$ and a catch-up program in the $14-26$ age group with an average coverage of $52\%$ and $73\%$. This program revealed an important herd effect \cite{ali2013genital}, so that vaccination decreased the incidence of genital warts (GW) even in the non-vaccinated men because of the protection of infection conferred by the vaccine, and the decreased transmission of the virus.

The model predicted a fast decline in the number of infections parallel to the decline in the number of GW in Australia with very similar values. However, this model was built with Spanish data on sexual behavior \cite{INE} and prevalence of HPV infection \cite{castellsague2012prevalence}, that might differ to the Australian one, and may explain the minor differences found between the model and the actual data published. Herd immunity in this model of STI is predicted much sooner than in other highly transmitted aerial transported infectious diseases as influenza or RSV, due to the structure of the network. This supports the need to build appropriate LSP networks. 

Other models have also predicted the protection of males by vaccinating girls and women, but only for men, as the model used by Bogaards et al. \cite{bogaards2015direct}. This model uses Bayesian techniques to study the herd immunity effect. However, in contrast with our model, it does not take into account the dynamics of the HPV transmission, the importance of age-groups and the different roles they play in the propagation of these viruses or the links among the MSM subpopulation and the heterosexual network. In this sense, a network model is required to study the impact of the vaccination strategies in short, medium and long time scales.

Vaccination strategies should seek an optimal effectiveness and efficiency. In this case, it can be seen the quick apparition of the herd immunity effect on males and females only vaccinating women. However, the herd immunity effect does not appear in  MSM. This can be the consequence of the large LSP numbers for MSM and their casual connections with women with large LSP numbers in the heterosexual subnetwork.

The model considers a quiet close community, where there is not much contact with other communities. This may not be the case in Spain which in $2016$ received over $75$ million tourists \cite{INEturismo}, representing almost the double of the number of Spanish inhabitants, and when sexual contacts are frequent. This may bias the results, as the herd immunity in Spain may not be so clear as in countries with less tourism.

Another issue that we must take into account, is the modelling of the population with a high number of contacts because these individuals are hubs in the network whose vaccination may induce a faster decline of the virus prevalence. Our approach is rather conservative in the assignment of LSP for men and women with $10$ or more links because we assume that all of them have similar LSP. However, it is expected that individuals with extreme values of LSP are favouring the transmission of HPV in such a way that  a targeted vaccination can show its benefit in a very short time.



\bibliographystyle{ieeetr}
\bibliography{bibtextbiblio}

\end{document}
