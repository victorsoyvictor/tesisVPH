\chapter{The effect of the Spanish \textit{Law of Political Parties} (LPP) on the attitude of the Basque Country population towards ETA}\label{paper2}

In Chapter \ref{paper1} we presented a first approach and detected several drawbacks. With the help of the expert we focus and improve the model. Also, we use other techniques to deal with the model uncertainty in order to avoid the detected inconveniences.

As we described in Chapter \ref{ETA}, in Jun 2002, the Spanish Government passed the "Law of Political Parties" (LPP) with the aim, among others, to prevent parties giving political support to terrorist organizations. This law affected the Basque nationalist party Batasuna, due to its proved relation with ETA. 

Along with that impact in the political arena, it is also reasonable to expect some impact in the sociological one. The question is: did the LPP have any effect on the attitude of the Basque Country population towards ETA? This question is particularly pertinent in light of the current situation in that region. On the one hand, generally speaking, it is well known that repressive initiatives taken by governments can generate sympathies, to some degree, towards the repressed organizations. On the other hand, in this particular case, violent activities carried out by ETA could be responsible for part of the population not expressing their political beliefs freely, so measures taken to prevent ETA's violence could encourage people to express themselves more openly. In other words, either has ETA together with its political wings, gained additional support from the population or has part of the Basque country been uninhibited because of that law? 

In order to study this problem, we use a new data set from the Euskobarometro survey \cite[Table 20]{eusko}, one of the best-known independent opinion polls in the region, which is periodically conducted by the University of the Basque Country. The period of time considered for this study is from the passing of the LPP (Jun 2002) to May 2005. The reason for this time limitation is that, in our opinion, during this period the anti-terrorist policies were reasonably homogeneous, while, from May 2005 on, a perceptible change occurred when the Congress approved the possibility the Government supports a dialogue process with ETA given that appropriate conditions to end violence occur \cite[Resoluci\'on 34, p. 13]{Congreso}. It is believed that this major event, and subsequent ones as well, could jeopardize the homogeneity necessary to conduct this study\footnote{A month later, ETA announced the cessation of its armed actions against the elected politicians in Spain, although later on it pointed out that this truce did not apply to members of the Government.}. Finally, remark that during the mentioned period of time, the 11-M bombing attacks in Madrid (on 11 March 2004), did not provoke major changes in the trend. We justified this in Chapter \ref{ETA}.

This chapter is organized as follows. In Section \ref{2.1}, data from Euskobarometro about the attitude of the Basque Country population towards ETA are retrieved and processed \cite[Table 20]{eusko}. Section \ref{2.2} is devoted to building a model describing the attitude dynamics towards ETA in the Basque Country. Model parameters are estimated in Section \ref{2.3} by fitting the model with the Euskobarometro data. In Section \ref{2.4} it is concluded that the LPP is responsible for an increasing attitude of rejection towards ETA and we quantify this effect by using a bootstrapping approach. Finally, some conclusions are drawn in Section \ref{2.5}.

\section{Data}\label{2.1}
In this case we are not going to use electoral data. Instead, we use data series from Euskobarometro. This was suggested by the expert because Euskobarometro appears every six months (is more regular) and it asks for a lot of relevant sociological and political questions in the Basque Country. Furthermore, when an individual vote to a political party, he/she does not necessarily assume and support all the ideas of the political party.
 
Thus, we have retrieved data series from the Euskobarometro of Nov 2010 on the attitude of the Basque Country population towards ETA \cite[Table 20]{eusko}. The eight types of attitudes towards ETA that appear in the Euskobarometro are as follows: Total support; Justification with criticism; Goals yes / Means no; Before yes / Not now; Indifferent; ETA scares; Total rejection; No answer. In order to simplify the model (the number of subpopulations) we group the eight attitudes in only three.

\begin{enumerate}
\item Support: people who have an attitude of support towards ETA. We consider the people with attitudes of "Total support" and "Justification with criticism" make up of this group.
\item Rejection: people who have an attitude of rejection against ETA. In this group we include the people with attitudes "Goals yes / Means no", "Before yes / Not now", "ETA scares" and "Total rejection". It could be questionable to include in this group the attitude "Goals yes / Means no", however, the fact is that there are parties and associations in the Basque Country with similar goals as ETA  and they have a rejection attitude towards ETA because of its violent means. 
\item Abstention: people who have no opinion or have an indifferent attitude towards ETA, that is, the "Indifferent" and the "No answer" groups.
\end{enumerate}

Data grouped in these three groups appear in Table \ref{2TABLA2} from May 1995 to May 2002 (before the passing of the LPP) and Table \ref{2TABLA3} for the period from Nov 2002 to May 2005 (after the passing of the LPP until the granting of permission by the Spanish Parliament to conduct a dialogue with ETA).

\begin{table}[ht]
\centering
\begin{tabular}{|l|c|c|c|}
\hline
  Survey date  & Support (\%) & Rejection (\%) & Abstention (\%) \\ 
\hline
 May 1995 & 7	&85	&8 \\
 Nov 1995 & 5	&87	&8 \\
 Nov 1996 & 6	&87	&7 \\
 Nov 1997 & 6	&86	&8 \\
 Nov 1998 & 5	&85	&10 \\
 May 1999 & 11	&76	&13 \\
 May 2000 & 8	&87	&5 \\
 Nov 2000 & 7	&87	&6 \\
 May 2001 & 3	&90	&7 \\
 Nov 2001 & 4	&88	&8 \\
 May 2002 & 2	&96	&2 \\
 \hline 
\end{tabular} 
\caption{Percentage of people in the Basque Country with respect to their attitude towards ETA from May 1995 to May 2002, when  the LPP was passed (pre-LPP scenario).}
\label{2TABLA2} 
\end{table}

\begin{table}[ht]
\centering
\begin{tabular}{|l|c|c|c|}
\hline
     Survey date  & Support (\%) & Rejection (\%) & Abstention (\%) \\ 
\hline
 Nov 2002 & 3	&	93	&	4 \\
 May 2003 & 2 &	95	&	3 \\
 Nov 2003 & 2	&	94	&	4 \\
 May 2004 & 3	&	93	&	4 \\
 Nov 2004 & 3	&	93	&	4 \\
 May 2005 & 2 &	93	&	5 \\
 \hline 
\end{tabular} 
\caption{Percentage of people in the Basque Country with respect to their attitude towards ETA from Nov 2002 to May 2005, after the passing of the LPP until the granting of the permission by the Spanish Parliament to conduct a dialogue with ETA (post-LPP scenario).}
\label{2TABLA3} 
\end{table}

The data in Table \ref{2TABLA2} will help us to estimate the parameters of the mathematical model. The data in Table \ref{2TABLA3} will be used to find out if the LPP affected the attitude of the people in the Basque Country towards ETA, and if so, quantify the effect of the LPP.

\section{Model building}\label{2.2}  
Bearing in mind Tables \ref{2TABLA2} and \ref{2TABLA3}, we distinguish three main different attitudes towards ETA and divide the population of the Basque Country into the following three subpopulations (time $t$ in years):

\begin{itemize}
\item $A_1(t)$, the percentage of people in the Basque Country who have an attitude of support towards ETA at time instant $t$,
\item $A_2(t)$ is the percentage of people who have an attitude of rejection towards ETA at time $t$, 
\item $A_3(t)$ corresponds to the percentage of population in the Basque Country whose attitude towards ETA is not defined, who abstain, or who simply do not want to state their opinion, at time $t$.
\end{itemize}

$A_1(t)$, $A_2(t)$ and $A_3(t)$ are the variables of the mathematical model. The assumptions used to build the equations of the model are as follows.

\begin{itemize}
\item A subpopulation $A_i$, whose people share a particular attitude towards a phenomenon, can influence the attitude of people of another subpopulation, $A_j$, towards the same phenomenon. This influence can be provoked either by direct contact, i.e., when people from $A_i$ and $A_j$ interact, or by indirect contact, i.e., through the interaction of a person in $A_i$ with his/her environment.
\item Regarding this latter way, in this context, it is assumed that the environment of a person in $A_j$ is made up of the flows and channels of information able to reach his/her sensorial system. Note that reaching a sensorial system does not imply necessarily reaching perception. Thus, alteration in that environment can provoke either changes in the attitude of that person in $A_j$ or not. Environment alteration can be provoked, in its turn, by the behaviour of people from the other subpopulations among other factors, attitude being itself considered as a part of that behaviour.
\item It is assumed that all people could access to all relevant information channels and flows, i.e., there is in principle a homogeneous environment affecting people of all the subpopulations. However, the interaction of a person with the environment varies on an individual basis, depending on both situational and non-situational factors. The individual initial attitude itself towards the subject of influence, for instance, is a non-situational factor which modulates environment influence, acting on that initial attitude either as an enabler or as a shield. 
\item It is not the goal of this work to clarify those factors of variation, but only to show the eventual changes in attitudes of the target populations and, if possible, to attribute those changes to the influence of other subpopulations, either directly or indirectly.  However, a diffuse idea about the involved processes, environment effectiveness differences etc., as a whole, can be obtained from the model. The non linear term $\beta_{ij} A_i A_j$ is the term that models these influences, it is the parameter $\beta_{ij}$ that, in some way, measures that environment effectiveness and includes the rest of the above-mentioned factors.
\end{itemize} 

The system of differential equations that models the evolution of attitudes towards ETA in Basque Country over time is given by

\begin{eqnarray}
A'_1(t) = &  (\beta_{21}  - \beta_{12}) A_2(t) A_1(t) + (\beta_{31}  - \beta_{13} ) A_3(t) A_1(t), \label{2eq1} \\
A'_2(t) = &  (\beta_{12}  - \beta_{21}) A_2(t) A_1(t) + (\beta_{32}  - \beta_{23} ) A_3(t) A_2(t), \\
A'_3(t) = &  (\beta_{13}  - \beta_{31}) A_3(t) A_1(t) + (\beta_{23}  - \beta_{32} ) A_3(t) A_2(t). \label{2eq3}                           
\end{eqnarray} 

The above system of differential equations can be represented by the diagram given as Figure \ref{2Modelo}.
 
\begin{figure}[h]
 \begin{center}
  \includegraphics[scale=0.7]{IMG/2Modelo.pdf}\\
  \caption{Graph depicting the model (\ref{2eq1})-(\ref{2eq3}). Circles are the subpopulations and arrows represent the flow of people who change their attitude towards ETA over time.}\label{2Modelo}
\end{center}
\end{figure} 

This new model has the advantge, if we compare it to the one in Chapter \ref{paper1}, that it is directly related to the opinion of the people, not via the political parties they vote.  

\section{Estimation of model parameters}\label{2.3}
The model has six unknown parameters $\beta_{ij}, i,j=1,2,3, i \neq j,$ and we should estimate them taking into account that the model has to be as close as possible to data in Table \ref{2TABLA2}, that is, before the passing of the LPP. 

To do that, we adapted the algorithm used in Chapter \ref{paper1}, implemented in \textit{Mathematica} \cite{Wolfram}, in order to compute the parameters which best fit the model with the data of Table \ref{2TABLA2} in the least square sense. The values of these parameters appear in Table \ref{2TABLA4}.

\begin{table}[ht]
\centering
\begin{tabular}{|cc|c|cc|}
\hline
Parameter & Value & & Parameter & Value \\ 
\hline
$\beta_{12}$  & $0.0815425$ & & $\beta_{21}$ & $0.0627668$ \\
$\beta_{13}$  & $0.000421055$ & & $\beta_{31}$ & $0.182483$ \\
$\beta_{23}$  & $0.0317568$ & & $\beta_{32}$ & $0.0216873$ \\
\hline 
\end{tabular}
\caption{Estimated model parameters.}
\label{2TABLA4} 
\end{table}

We can see the fitting graphically in the Figure \ref{2Ajuste}.

\begin{figure}[h]
 \begin{center}
  \includegraphics[scale=0.38]{IMG/2GraficaAjuste.pdf}\\
  \caption{Graph representing the fitting. The lines are the corresponding model functions and the points are data from Table \ref{2TABLA2}. Support subpopulation $A_1(t)$ on the left, Rejection subpopulation $A_2(t)$ in the middle, and Abstention subpopulation $A_3(t)$ on the right.} \label{2Ajuste}
\end{center}
\end{figure} 

\section{Analysis of the effect of the LPP}\label{2.4}
In Figure \ref{2Pre}, we can see the model predictions (line) for every subpopulation after LPP passing (Jun 2002) until May 2005 and data from Table \ref{2TABLA3} (points).

\begin{figure}[h]
 \begin{center}
  \includegraphics[scale=0.38]{IMG/2PreBootstrapping.pdf}\\
  \caption{Graph of model prediction after LPP (Jun 2002) until May 2005 (line) with data from Table \ref{2TABLA3} (points). The question is if the differences are due to model-fitting errors (non-significant) or are due to the LPP (significant).} \label{2Pre}
\end{center}
\end{figure} 

Looking at the graph (Figure \ref{2Pre}), it is difficult to say if the differences between the points and the model prediction, on one hand, are attributable to model-fitting errors, i.e., the differences are non-significant and consequently the LPP did not have effect on the general attitude towards ETA, or, on the other hand, if the differences are significant and attributable to the LPP.

\subsection{Finding out if the differences between the data and the model prediction are (or are not) due to the effect of the LPP on the Basque population?}
An uncertainty study of the predictions of the model will allow us to determine if differences between the data and the model prediction are significant. Thus, in order to obtain more information on the output of the mathematical model, we use a residual bootstrapping approach. Considering the general procedure presented by Dogan in \cite{Dogan}, we study error terms for the estimated parameters and resample these terms using bootstrapping. Then, we obtain new perturbed data by adding the resampled error to Table \ref{2TABLA2} data. For each new data perturbation calculated, we compute the parameters that best fit the model with the perturbed dataset. Once we compute the set of parameter values obtained by fitting the model with the perturbed data, we solve the model with these parameters and compute the outputs in the required time instants. Taking the $90\%$ confidence interval of each output from each subpopulation by percentile $5$ and percentile $95$ and comparing with the corresponding datum from Table \ref{2TABLA3}, i.e., if the datum lies inside the confidence interval or not, we will be able to conclude if the LPP had effect on the attitude of Basque population towards ETA or not, and to quantify it by measuring the distance of the datum to the extremes of the confidence interval.

\subsection{Error term analysis}   
First, we compute the output of the model with the parameters in Table \ref{2TABLA4} in the time instants appearing in Table \ref{2TABLA2} (from May 2005 to May 2002) and compute their differences with the corresponding data from Table \ref{2TABLA2}. The results can be seen in Table \ref{2TABLAERRORES}.

\begin{table}[ht]
\centering
\begin{tabular}{|cccc|}
\hline
  Survey date  & $A_1(t)-\hat{A}_1(t)$ & $A_2(t)-\hat{A}_2(t)$ & $A_3(t)-\hat{A}_3(t)$ \\   
\hline
 Nov 1995 (t=1)  &-1.060970135& -0.172986635& 1.23395677	\\
 Nov 1996 (t=2)  &2.216235209 &	-2.382102564& 0.165867355\\
 Nov 1997 (t=3)  & 2.992820716&	-1.740283841& -1.252536874\\
 Nov 1998 (t=4)  & 0.792540417&	1.008587701& -1.801128118 \\
 May 1999 (t=5)   & 5.386657918&	-6.874217326& 1.487559407\\
 May 2000 (t=6)   & 1.02135272&1.909570229	& -2.930922949\\
 Nov 2000 (t=7)  & 0.993820922&  -0.260804633 & -0.733016289\\
 May 2001 (t=8)   & -1.762253593&	1.194714054& 0.567539539\\
 Nov 2001 (t=9)  & 0.250086725&	-1.385370292	& 1.135283566\\
 May 2002 (t=10) & -1.164726381	&	7.035272584& -5.870546202\\    
 \hline 
\end{tabular} 
\caption{Residual or error terms. $A_i(t)$ are the real data (Table \ref{2TABLA2}) and $\hat{A}_i(t)$ are the predictions of the model.} 
\label{2TABLAERRORES} 
\end{table}   

Now, we analyse whether the error terms $e_{1t} = A_1(t)-\hat{A}_1(t)$, $e_{2t} = A_2(t)-\hat{A}_2(t)$ and $e_{3t} = A_3(t)-\hat{A}_3(t)$ are correlated. The Pearson correlation coefficient is used, and the results obtained are as follows: $\rho_{12}=-0.782$, $p-value=0.007$; $\rho_{13}=0.270$, $p-value=0.4514$; $\rho_{23}=-0.811$, $p-value=0.004$. Note that $\rho_{ij}$ is the Pearson correlation coefficient between $e_{it}$ and $e_{jt}$. Therefore, there is dependence between the errors.

Taking into account runs test, we also study if each error term is autocorrelated. Note that this non-parametric test can be used to check the hypothesis that the elements of a sequence are mutually independent. In this case, the results are as follows: $z_1=1.677$, $p-value=0.094$; $z_2=-0.335$, $p-value=0.737$; and $z_3=0.000$, $p-value=1.000$. None of the test statistic values is statistically significant ($p-value>0.05$); therefore the claim that there is autocorrelation should be rejected. $z_i$ is the runs test statistic value for each case. 

Additionally, the normality of the distribution of errors is determined by using non-parametric tests. A goodness-of-fit analysis suggests that each error term is normally distributed. The Kolmogorov-Smirnov and Shapiro-Wilk tests have $p-values$ of $0.200$, $0.200$, $0.200$ and $0.560$, $0.552$, $0.154$, respectively. Moreover, Mardia's multivariate normality test is applied to the sample ($e_{1t}$, $e_{2t}$), $t= 0,1, \dots,10$ (see Table \ref{2TABLAERRORES}). In this case, Mardia's test has a $p-value$ equal to $0.282$ ($p-value>0.05$). Therefore, we can accept that vector ($e_{1t}$, $e_{2t}$) presents a bivariate normal distribution. To be precise, we accept that

\begin{equation}
\centering
(e_{1t}, e_{2t}) \sim N_2 
\left[
\left( 
\begin{array}{cc}
\mu_{e_{1t}}  \\
\mu_{e_{2t}}   \\
\end{array} 
\right),  
\left( 
\begin{array}{cc}
\sigma^2_{e_{1t}} & \rho_{12}\sigma_{e_{1t}}\sigma_{e_{2t}}  \\
 \rho_{12}\sigma_{e_{1t}}\sigma_{e_{2t}}  & \sigma^2_{e_{2t}} \\
\end{array} 
\right)
\right],
\label{2distribucionnormal}
\end{equation}

where $\mu_{e_{it}}$ and $\sigma_{e_{it}}$, $i=1, 2$, are the mean and the standard deviation of $e_{it}$, respectively, and $\rho_{12}$ is the Pearson correlation coefficient between $e_{1t}$ and $e_{2t}$. These parameters can be estimated using the errors in Table \ref{2TABLAERRORES} and the values are $\mu_{e_{1t}}=0.966556$, $\mu_{e_{2t}}=-0.166762$, $\sigma_{e_{1t}}=2.15643$, $\sigma_{e_{2t}}=3.54782$ and $\rho_{12}=-0.738104$. Finally, considering that $e_{1t}+e_{2t}+e_{3t}=0$, $t=1,\dots,10$, $e_{3t}$ can be calculated by $e_{3t}=-e_{1t}-e_{2t}$. $e_{1t}$ and $e_{2t}$ are estimated by (\ref{2distribucionnormal}).   

\subsection{Generating new perturbed data} 
Bearing in mind data from Table \ref{2TABLA2} (pre-LPP data), for $t=$ May 1995, Nov 1995, $\ldots$,  May 2002, we generate $10$ random pairs $(e_{1t}, e_{2t})$ following the multivariate distribution given by the expression (\ref{2distribucionnormal}) and $e_{3t}$ as $e_{3t}=-e_{1t}-e_{2t}$. Thus, we have $10$ vectors $(e_{1t}, e_{2t}, e_{3t})$ for $t=$ May 1995, Nov 1995, $\ldots$,  May 2002, and we add them to data in Table \ref{2TABLA2}, obtaining a new set of perturbed data. Then, we compute the parameters which best fit the model with the new set of perturbed data in the least square sense and store them, using the same procedure we used to estimate the parameters of Table \ref{2TABLA4}.

We repeat this procedure $5000$ times in order to obtain $5000$ sets of parameters that fit each set of perturbed data (pre-LPP data plus $(e_{1t}, e_{2t}, e_{3t})$ for each $t$).

\subsection{Obtaining confidence intervals for model outputs}     
For each one of the $5000$ set of parameters, we solve the system of differential equations (\ref{2eq1})-(\ref{2eq3}) and compute the output of the solution, i.e., in the three subpopulations $A_1(t)$, $A_2(t)$ and $A_3(t)$, for $t=$ Nov 2002, May 2003, Nov 2003, May 2004, Nov 2004 and May 2005 (post-LPP data). Thus, for each $t$, and for each subpopulation, we have a set of $5000$ model output values. Then, we compute the mean and the $90\%$ confidence interval (CI) by percentiles $5$ and $95$. The results obtained can be seen in Table \ref{2TABLACI}.

\begin{table}[ht]
\centering
\begin{scriptsize}
\begin{tabular}{|c|cc|cc|cc|}
\hline
& \multicolumn{2}{|c|}{Support} & \multicolumn{2}{|c|}{Rejection} & \multicolumn{2}{|c|}{Abstention}\\
\hline
 & Mean & $90\%$ CI  & Mean & $90\%$ CI & Mean & $90\%$ CI \\ 
\hline
Nov 2002	& $	4.595	$ & $[	2.387,	7.978	]$ & $	86.643	$ & $[	83.893	,	89.018	]$ & $	8.762	$ & $[	7.031,	10.831	]$ \\
May 2003	& $	5.310	$ & $[	2.673,	9.912	]$ & $	85.144	$ & $[	82.737	,	87.470	]$ & $	9.546	$ & $[	6.546,	12.453	]$ \\
Nov 2003	& $	6.161	$ & $[	3.421,	10.514	]$ & $	84.180	$ & $[	81.800	,	86.306	]$ & $	9.660	$ & $[	4.753,	13.634	]$ \\
May 2004	& $	6.791	$ & $[	4.549,	9.627	]$ & $	84.099	$ & $[	80.615	,	87.984	]$ & $	9.110	$ & $[	3.774,	13.699	]$ \\
Nov 2004	& $	7.021	$ & $[	5.673,	8.509	]$ & $	84.839	$ & $[	80.374	,	89.566	]$ & $	8.141	$ & $[	3.577,	12.839	]$ \\
May 2005	& $	6.755	$ & $[	5.117,	8.218	]$ & $	85.967	$ & $[	80.931	,	90.410	]$ & $	7.278	$ & $[	3.949,	11.225	]$ \\
\hline 
\end{tabular}
\end{scriptsize}
\caption{Means and $90\%$ confidence intervals of the model output. We estimate these predictions (point prediction and interval prediction) solving the model (\ref{2eq1})-(\ref{2eq3}) for each one of the $5000$ sets of parameters calculated by fitting the model with the perturbed pre-LPP data.}
\label{2TABLACI} 
\end{table}

In Figure \ref{2gCI} we can see graphically, for each subpopulation, the data from Table \ref{2TABLA2} to Table \ref{2TABLA3} (points), the deterministic model prediction (line) and the $90\%$ confidence intervals (error bars). The points in the middle of the confidence intervals are the means of $5000$ outputs for each subpopulation and each time instant where we have data about attitudes towards ETA. These mean values are those appearing in Table \ref{2TABLACI}. 

\begin{figure}[h]
 \begin{center}
  \includegraphics[scale=0.9]{IMG/2GraficaFinal.pdf}\\
  \caption{In this graph we show the attitude towards ETA (points), the model deterministic prediction (line) and the error bars corresponding to $90\%$ confidence intervals in the same time instants as we have data in Tables \ref{2TABLA2} and \ref{2TABLA3}. The points inside the confidence intervals are the mean of the $5000$ outputs for every subpopulation in every time instant. The vertical axis is placed on the time instant when the LPP was passed (Jun 2002).} \label{2gCI}
\end{center}
\end{figure} 

If we observe the right-hand side of the vertical axis in the three graphs, we realise two facts: on the one hand, there are differences between the deterministic model predictions and the means of Table \ref{2TABLACI}. These differences indicate that the model is sensitive to parameter changes; on the other hand, most of the attitude prevalence points lie out of their corresponding $90\%$ confidence intervals, and those that lie inside are placed in the interval extremes. This leads us to say that the LPP had effect on the attitude of the Basque population towards ETA. Moreover, we can see that, around the 11-M bombing attacks in Madrid, the points still lie outside of the confidence intervals, and this fact leads us to give another reason on that the Madrid attacks had hardly any effect on the general attitude of the Basque Country population towards ETA.   

In Table \ref{2TABLAEfecto}, we show the differences between the attitude data (Table \ref{2TABLA2}) and their corresponding $90\%$ interval extremes (Table \ref{2TABLACI}), in order to obtain an upper and lower bound measurement of the LPP effect on the Basque population.

\begin{table}[ht]
\centering
\begin{tabular}{|c|ccc|}
\hline
 & Support & Rejection & Abstention \\ 
\hline
Nov 2002	& $[	0.39	,	5.98	]$ & $[	6.98	,	12.11	]$ & $[	5.03	,	8.83	]$ \\
May 2003	& $[	-0.33,	6.91	]$ & $[	5.53	,	10.26	]$ & $[	2.55	,	8.45	]$ \\
Nov 2003	& $[	1.42	,	8.51	]$ & $[	8.69	,	13.20	]$ & $[	1.75	,	10.63	]$ \\
May 2004	& $[	2.55	,	7.63	]$ & $[	6.02	,	13.38	]$ & $[	-0.23,	9.70	]$ \\
Nov 2004	& $[	2.67	,	5.51	]$ & $[	3.43	,	12.63	]$ & $[	-0.42,	8.84	]$ \\
May 2005	& $[	2.12	,	5.22	]$ & $[	2.59	,	12.07	]$ & $[	-0.05,	7.22	]$ \\
\hline 
\end{tabular}
\caption{Distances between Table \ref{2TABLA3} data and the extremes of their corresponding $90\%$ confidence intervals. Intervals with negative values mean that the attitude prevalence datum lies inside the $90\%$ confidence interval.}
\label{2TABLAEfecto} 
\end{table}

Looking at Figure \ref{2gCI} and Table \ref{2TABLAEfecto}, we can conclude that the LPP had an effect on increasing the number of people who have an attitude of rejection towards ETA at the expense to the ones who previously have an attitude of support or abstention. Moreover, the increase is strong until Nov 2003 - May 2004 (minimum of $8.69\%$ and maximum of $13.38\%$), when, even maintaining values greater than before the law was passed, the trend starts to decrease slightly. 

\section{Conclusion}\label{2.5}  
In this chapter, we present a mathematical model to study the evolution dynamics of the attitude of Basque population towards ETA. Once the model is stated, we determine the model parameters in such a way that the model is able to fit the data from Table \ref{2TABLA2}. 

Then, we use it to find out if the LPP had any effect on changing the attitude of the Basque population towards ETA over the time after its passing. To do that, we use a residual bootstrapping approach to get more information about the estimated parameter values and to obtain output model values. With these outputs, we calculated confidence intervals that allowed us to determine if the differences between Table \ref{2TABLA3} data and model outputs are related to intrinsic model errors or LPP's effect. The use of a residual bootstrapping technique has improved the model uncertainty treatment. However, we must say that during its application we realised that, to be applied, it has to fulfill restrictive hypotheses and it does not consider the uncertainty in the initial condition. Therefore, we will attempt to avoid the mentioned inconveniences in the next chapter.

As we can see from the results, there is a clear effect of the LPP in the time interval May 2002 until May 2005, where the Rejection attitude increases strongly until Nov 2003 - May 2004 (minimum of $8.69\%$ and maximum of $13.38\%$) and then, a slight decrease, maintaining values greater than those before the passing of the LPP during the whole period. These greater values of the Rejection subpopulation are at the expense of the other subpopulations, Support and Abstention. The 11-M bombing attacks in Madrid, in Mar 2004, could have had some local impact in the polls, but they do not interfere in the general trend.

Therefore, the effect of the LPP, for Rejection subpopulation, can be measured from May 2002 to  May 2005 as an increase of $2.59\%$ in the lower case and an increase of $13.38\%$ in the higher case of the people changing to a rejection attitude towards ETA (see Table \ref{2TABLAEfecto}).

Note that, in this chapter, we have stated an improved model and the uncertainty has been studied using a bootstrapping technique and we could to quantify the effect of the LPP.

Finally, as we mentioned in Chapter \ref{CAPINTRO}, nowadays, there are important political parties, mass media and organizations supporting the negotiation with ETA as the best choice to finish with its activities. However, this chapter supports that the opposite idea may be possible, that is, the appropriate use of the laws is an useful tool to fight against these extreme organizations and change the citizen's mind against their social pressure. 

