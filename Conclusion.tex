\chapter{Conclusion}\label{conclusion}

In this dissertation, we focus our study in the especial situation occurring in the Basque Country where the Basque revolutionary organisation ETA has been using different forms of terrorism to achieve its political goals. Popular support is an important enabler for radical violent organizations and it may be crucial for their survival. At the same time, extremist groups have also an impact in the societies where they are inserted, especially if those groups are engaged in violent activities. Social and behavioral scientists try to find clues about how that interaction may affect those people, either at the group or at the individual level, in order to foresee subsequent dynamics. 

To do that, we propose mathematical models to analyse the dynamics of the attitude the Basque Country population have towards ETA, its goals and the means it uses to achieve them. These models allow us to know if there will be a reduction in the population of ETA's supporters, the main source of ETA members, or not. Taking into account the unpredictability of the human behavior, errors in data, etc., it is necessary to introduce the treatment of the uncertainty in the proposed models. Modelling and treatment of the uncertainty are the main threads across the chapters of the present dissertation, on the one hand stating a reliable and well supported model and, on the other hand, using known techniques to deal with uncertainty and proposing new ones. 

As a result of the work done with the above goals in mind, in the following, we point out the main general conclusions of this dissertation:

\begin{enumerate}
\item Under the Socio-Political point of view:
	\begin{itemize}
	\item The "Law of Political Parties" produced, from May 2002 to  May 2005, an increase of $2.59\%$ in the lower case and an increase of $13.38\%$ in the higher case of the people with a rejection attitude towards ETA. Therefore, even though there are important political parties, mass media and organizations supporting the negotiation with ETA as the best choice to finish with its activities, the appropriate use of the laws is an useful tool to fight against these extreme organizations and change the citizen's mind against their social pressure.
	\item Prediction figures indicate stabilization in the evolution of the attitudes towards ETA over the next few years, and therefore stabilization in a hypothetical pool of candidates willing to join the organization in upcoming years. As a result, the presented prediction states that the popular support to the ETA will remain stable, if and when the current scenario does not change, and around $3\%$ (with a minimum of $1.41\%$ and a maximum of $4.54\%$) of the Basque citizens will support ETA unconditionally over the next four years. 
	\end{itemize}

\item Under the Mathematical point of view:
	\begin{itemize}
	\item We have helped to introduce dynamic mathematical models in a complex area of Social Sciences as is the ideological or opinion dynamic evolution.
	\item The introduction of these dynamical models has been carried out using real data with short and medium term forecasting, not via the classical approaches using dynamic analysis techniques.
	\item The relation with the expert has been crucial in this dissertation development to address properly this research. This relation provided the possibility of developing interesting research worthy in both areas, Social Sciences and Mathematics. 
	\item We have provided a useful example of dynamic models based on differential equations to quantify, \textit{a posteriori}, the effect of government laws.
	\item We have provided some advances in the introduction of uncertainty in dynamic models. To be precise, in Chapter \ref{paper3} we have proposed a computational technique to deal with the uncertainty from the very beginning facing relevant problems as the knowledge of the probability distribution of the parameters and initial conditions and saving computational issues appearing when we tackle realistic models with a high number of equations and parameters. 
	\end{itemize}
\end{enumerate}

As a final conclusion, we would like to say that, under our point of view, uncertainty treatment should be considered as an important part of the model building and analysis.