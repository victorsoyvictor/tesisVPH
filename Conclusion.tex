\chapter{Conclusions}
The random network of sexually transmitted HPV including up to $100,000$ nodes, was developed to fit the data of surveys concerning the number of sexual partners throughout life \cite{Acedo2017,DezDomingo2017}. Standard continuous models are insufficient to accurately predict transmission because they do not account for the individual to individual transmission of the infection, the role of hubs in disseminating the virus through the rest of the population and neither the vaccination campaigns targeting specific groups of individuals.

This network has successfully been applied to the stable state of infections by LR and HR HPV genotypes in Spain  \cite{Acedo2017}. In this study we mimicked the results found in the HPV vaccination campaign in Australia \cite{ali2013genital}, and showed very reliable results. 

Models based upon continuous differential equations predict a slower decrease in the number of infected individuals after implementing similar vaccination campaigns \cite{elbasha2007model}. Hence, the case of the HPV vaccination in Australia provides one of the best real scenarios for testing new network models in mathematical epidemiology. There is an on-going debate on the pertinence of an approach based upon networks on epidemiology \cite{Eubank} and this work contributes to show the necessity of such an approach in many cases, in particular, in those corresponding to STI.

To validate the model, we used the Australian experience, with two different vaccination coverages: routinely vaccination campaign for $12-13$ year-old girls with a coverage of $73\%$ and $83\%$ and a catch-up program in the $14-26$ age group with an average coverage of $52\%$ and $73\%$. This program revealed an important herd effect \cite{ali2013genital}, so that vaccination decreased the incidence of genital warts (GW) even in the non-vaccinated men because of the protection of infection conferred by the vaccine, and the decreased transmission of the virus.

The model predicted a fast decline in the number of infections parallel to the decline in the number of GW in Australia with very similar values. However, this model was built with Spanish data on sexual behavior \cite{INE} and prevalence of HPV infection \cite{castellsague2012prevalence}, that might differ to the Australian one, and may explain the minor differences found between the model and the actual data published. Herd immunity in this model of STI is predicted much sooner than in other highly transmitted aerial transported infectious diseases as influenza or RSV, due to the structure of the network. This supports the need to build appropriate LSP networks. 

Other models have also predicted the protection of males by vaccinating girls and women, but only for men, as the model used by Bogaards et al. \cite{bogaards2015direct}. This model uses Bayesian techniques to study the herd immunity effect. However, in contrast with our model, it does not take into account the dynamics of the HPV transmission, the importance of age-groups and the different roles they play in the propagation of these viruses or the links among the MSM subpopulation and the heterosexual network. In this sense, a network model is required to study the impact of the vaccination strategies in short, medium and long time scales.

Vaccination strategies should seek an optimal effectiveness and efficiency. In this case, it can be seen the quick apparition of the herd immunity effect on males and females only vaccinating women. However, the herd immunity effect does not appear in  MSM. This can be the consequence of the large LSP numbers for MSM and their casual connections with women with large LSP numbers in the heterosexual subnetwork.

The model considers a quiet close community, where there is not much contact with other communities. This may not be the case in Spain which in $2016$ received over $75$ million tourists \cite{INEturismo}, representing almost the double of the number of Spanish inhabitants, and when sexual contacts are frequent. This may bias the results, as the herd immunity in Spain may not be so clear as in countries with less tourism.

Another issue that we must take into account, is the modelling of the population with a high number of contacts because these individuals are hubs in the network whose vaccination may induce a faster decline of the virus prevalence. Our approach is rather conservative in the assignment of LSP for men and women with $10$ or more links because we assume that all of them have similar LSP. However, it is expected that individuals with extreme values of LSP are favouring the transmission of HPV in such a way that  a targeted vaccination can show its benefit in a very short time.
