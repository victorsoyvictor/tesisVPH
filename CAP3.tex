\chapter{A probabilistic estimation and prediction technique for the evolution of the attitude of the Basque Country population towards ETA}\label{paper3}

This chapter is twofold: on the one hand we are going to use the model developed in Chapter \ref{paper2} to predict the evolution dynamics of the attitude of Basque population towards ETA over the next few years in order to know if the Supporters, the source of ETA members, decrease or not; on the other hand, we introduce a new technique to deal with uncertainty, avoiding some inconveniences detected in the previous chapters, that allows us to give the predictions using confidence intervals.  

Thus, here we propose a computational approach where the data, retrieved from surveys, play a fundamental role to introduce the uncertainty, in estimation and prediction, from the very beginning. 

The chapter is organized as follows. In Section \ref{3.1}, we summarize the model building presented in Chapter \ref{paper2} introducing some variations and the Euskobarometro data since May 2005. In Section \ref{3.2} we propose a technique which will allow us to obtain a set of model parameters that provide $95\%$ confidence intervals for each time instant such that the data uncertainty is captured. We will call this technique \textit{probabilistic estimation}. With the set of parameters obtained in Section \ref{3.2}, in Section \ref{3.3} we obtain a \textit{probabilistic prediction} of the attitude towards ETA of the people of the Basque Country over the next four years. In Section \ref{3.5}, we discuss the results and present the conclusion.

\section{Data and model}\label{3.1}
In Section \ref{2.1} we retrieved data series from the Euskobarometro of November 2012 on the attitude of the Basque Country population towards ETA \cite[Table 20]{eusko}. We gathered the eight different attitudes in only three as we did in the Chapter \ref{paper2}. Data grouped in these three groups appear, in percentages, in Figures \ref{datosS}, \ref{datosR} and \ref{datosA} from May 1995 until Nov 2012. In May 2005 the Spanish Parliament approved the possibility the Government to support dialogue with ETA, what has been considered as a substantial change in the anti-terrorist policy. This policy is still in force and it justifies that we choose this time instant as our model's initial condition. In Table \ref{c3TABLA1} we present the figures in percentages of each subpopulation from May 2005 to Nov 2012.

\begin{table}[h]
\centering
\begin{tabular}{|l|c|c|c|}
\hline
  Survey date  & Support (\%) & Rejection (\%) & Abstention (\%) \\ 
\hline
May 05	&	2	&	93	&	5	\\
Nov 05	&	3	&	93	&	4	\\
May 06	&	3	&	93	&	4	\\
Nov 06	&	4	&	86	&	10	\\
May 07	&	2	&	84	&	14	\\
Nov 07	&	2	&	90	&	8	\\
May 08	&	3	&	90	&	7	\\
Nov 08	&	1	&	93	&	6	\\
May 09	&	4	&	90	&	6	\\
Nov 09	&	3	&	89	&	8	\\
May 10	&	3	&	90	&	7	\\
Nov 10	&	4	&	88	&	8	\\
May 11	&	4	&	90	&	6	\\
Nov 11	&	3	&	89	&	8	\\
May 12	&	5	&	89	&	6	\\
Nov 12	&	3	&	92	&	5	\\
 \hline 
\end{tabular} 
\caption{Percentage of people in the Basque Country with respect to their attitude towards ETA from May 2005 to Nov 2012.}
\label{c3TABLA1} 
\end{table}

In Section \ref{2.2}, we introduced the following system of nonlinear differential equations to describe the evolution of the attitudes towards ETA in the Basque Country over time:

\begin{eqnarray}
A'_1(t) = &  \beta_{21} A_2(t) A_1(t) - \beta_{12} A_1(t) A_2(t) + \beta_{31} A_3(t) A_1(t) - \beta_{13} A_1(t) A_3(t), \nonumber \\
A'_2(t) = &  \beta_{12} A_1(t) A_2(t) - \beta_{21} A_2(t) A_1(t) + \beta_{32} A_3(t) A_2(t) - \beta_{23} A_2(t) A_3(t), \nonumber \\
A'_3(t) = &  \beta_{13} A_1(t) A_3(t) - \beta_{31} A_3(t) A_1(t) + \beta_{23} A_2(t) A_3(t) - \beta_{32} A_3(t) A_2(t). \nonumber
\end{eqnarray} 

Taking $\gamma_{12} = \beta_{12}  - \beta_{21}$, $\gamma_{13} = \beta_{13}  - \beta_{31}$ and $\gamma_{23} = \beta_{23}  - \beta_{32}$, the above system can be simplified as follows

\begin{eqnarray}
A'_1(t) = &  -\gamma_{12} A_2(t) A_1(t) - \gamma_{13} A_3(t) A_1(t), \label{3eq1} \\
A'_2(t) = &   \gamma_{12} A_2(t) A_1(t) - \gamma_{23} A_3(t) A_2(t), \\
A'_3(t) = &   \gamma_{13} A_3(t) A_1(t) + \gamma_{23} A_3(t) A_2(t). \label{3eq3}                           
\end{eqnarray} 

Note that if $\gamma_{ij} > 0$ the net movement of individuals is from $A_i$ to $A_j$. The above system of differential equations can be represented by the diagram of Figure \ref{3Modelo}. 
 
\begin{figure}[h]
 \begin{center}
  \includegraphics[scale=0.7]{IMG/3Modelo2.pdf}\\
  \caption{Graph depicting the model (\ref{3eq1})-(\ref{3eq3}). The circles are the subpopulations and the arrows represent the flow of people who change their attitude towards ETA over the time.}\label{3Modelo}
\end{center}
\end{figure} 

\section{Probabilistic estimation: A computational technique to determine the empirical probabilistic distribution of model parameters}\label{3.2}

Roughly, this technique can be summarised in the following steps:

\begin{enumerate}
\item Assuming a probability distribution for every survey, determine data survey confidence intervals.
\item Use the survey probability distributions to sample data (survey simulations) a large number of times, and we obtain the model parameters that make the model fits the sampled data.
\item Among the obtained sets of model parameters, select the ones for which the model output confidence intervals computed substituting all of them into the model and solving it, are as close as possible of the data survey confidence intervals.
\end{enumerate}

Now, we are going to describe it in detail.

\subsection{Data}
Data in Table \ref{c3TABLA1} correspond to the mean percentage obtained from the Euskobarometro surveys since May 2005 to Nov 2012 \cite[Table 20]{eusko}. In the technical specifications of each survey we can see sample sizes of $1800$ and $1200$ interviews (see column 3 in Table \ref{c3TABLA2}). 

Taking into account that the sample is not the same for each survey, let us assume that the survey outputs are independent. For each one of the $16$ available surveys, let us denote by $X^j=(X_1^j,X_2^j,X_3^j)$, $0\leq X_i^j\leq n_j$, $i=1,2,3$, $j=1,\ldots,16$, a random vector whose entries are $X_1^j = $ Support, $X_2^j = $ Rejection, $X_3^j = $ Abstention and $n_j \in \{1200,1800\}$ is the sample size of survey $j$. These components represent exclusive selections (events) with probabilities

\[
P^j(X_1^j=x_1) = \theta_1^j, P^j(X_2^j=x_2) = \theta_2^j, P^j(X_3^j=x_3) = \theta_3^j, \ j=1,\ldots,16,
\]

where $\theta_1^j$, $\theta_2^j$ and $\theta_3^j$ are the percentages collected in Table \ref{c3TABLA1} for each survey $j$, $j=1,\ldots,16$. We have accepted that each random vector $X^j$ follows a multinomial (trinomial) probability distribution. Therefore, the probability that $X_1^j$ occurs $x_1$ times, $X_2^j$ occurs $x_2$ times and $X_3^j$ occurs $x_3$ times is given by

\[
P_{n_j}^j(x_1,x_2,x_3) = \frac{n_j!}{x_1! x_2! x_3!} (\theta_1^j)^{x_1} (\theta_2^j)^{x_2} (\theta_3^j)^{x_3}, \; j=1,\ldots, 16, 
\]

where $x_1+x_2+x_3=n_j$. The resulting trinomials for each Euskobarometro survey can be seen in the column 4 of Table \ref{c3TABLA2}.

\begin{table}[h]
\centering
\begin{scriptsize}
\begin{tabular}{|c|c|c|c|}
\hline
 & Survey  &Sample &	 Joint trinomial probability function	\\
 & dates	& size	&   \\
\hline
$j=1$	&	$t_{1}=$ May 05	&	$n_{1}=1800$	&	$P_{1800}^{1}(x_1,x_2,x_3) = \frac{1800!}{x_1! x_2! x_3!}0.02^{x_1}0.93^{x_2}0.05^{x_3}$	\\
$j=2$	&	$t_{2}=$ Nov 05	&	$n_{2}=1200$	&	$P_{1200}^{2}(x_1,x_2,x_3) = \frac{1200!}{x_1! x_2! x_3!}0.03^{x_1}0.93^{x_2}0.04^{x_3}$	\\
$j=3$	&	$t_{3}=$ May 06	&	$n_{3}=1800$	&	$P_{1800}^{3}(x_1,x_2,x_3) = \frac{1800!}{x_1! x_2! x_3!}0.03^{x_1}0.93^{x_2}0.04^{x_3}$	\\
$j=4$	&	$t_{4}=$ Nov 06	&	$n_{4}=1200$	&	$P_{1200}^{4}(x_1,x_2,x_3) = \frac{1200!}{x_1! x_2! x_3!}0.04^{x_1}0.86^{x_2}0.1^{x_3}$	\\
$j=5$	&	$t_{5}=$ May 07	&	$n_{5}=1200$	&	$P_{1200}^{5}(x_1,x_2,x_3) = \frac{1200!}{x_1! x_2! x_3!}0.02^{x_1}0.84^{x_2}0.14^{x_3}$	\\
$j=6$	&	$t_{6}=$ Nov 07	&	$n_{6}=1200$	&	$P_{1200}^{6}(x_1,x_2,x_3) = \frac{1200!}{x_1! x_2! x_3!}0.02^{x_1}0.9^{x_2}0.08^{x_3}$	\\
$j=7$	&	$t_{7}=$ May 08	&	$n_{7}=1800$	&	$P_{1800}^{7}(x_1,x_2,x_3) = \frac{1800!}{x_1! x_2! x_3!}0.03^{x_1}0.9^{x_2}0.07^{x_3}$	\\
$j=8$	&	$t_{8}=$ Nov 08	&	$n_{8}=1200$	&	$P_{1200}^{8}(x_1,x_2,x_3) = \frac{1200!}{x_1! x_2! x_3!}0.01^{x_1}0.93^{x_2}0.06^{x_3}$	\\
$j=9$	&	$t_{9}=$ May 09	&	$n_{9}=1200$	&	$P_{1200}^{9}(x_1,x_2,x_3) = \frac{1200!}{x_1! x_2! x_3!}0.04^{x_1}0.9^{x_2}0.06^{x_3}$	\\
$j=10$	&	$t_{10}=$ Nov 09	&	$n_{10}=1200$	&	$P_{1200}^{10}(x_1,x_2,x_3) = \frac{1200!}{x_1! x_2! x_3!}0.03^{x_1}0.89^{x_2}0.08^{x_3}$	\\
$j=11$	&	$t_{11}=$ May 10	&	$n_{11}=1200$	&	$P_{1200}^{11}(x_1,x_2,x_3) = \frac{1200!}{x_1! x_2! x_3!}0.03^{x_1}0.9^{x_2}0.07^{x_3}$	\\
$j=12$	&	$t_{12}=$ Nov 10	&	$n_{12}=1200$	&	$P_{1200}^{12}(x_1,x_2,x_3) = \frac{1200!}{x_1! x_2! x_3!}0.04^{x_1}0.88^{x_2}0.08^{x_3}$	\\
$j=13$	&	$t_{13}=$ May 11	&	$n_{13}=1200$	&	$P_{1200}^{13}(x_1,x_2,x_3) = \frac{1200!}{x_1! x_2! x_3!}0.04^{x_1}0.9^{x_2}0.06^{x_3}$	\\
$j=14$	&	$t_{14}=$ Nov 11	&	$n_{14}=1200$	&	$P_{1200}^{14}(x_1,x_2,x_3) = \frac{1200!}{x_1! x_2! x_3!}0.03^{x_1}0.89^{x_2}0.08^{x_3}$	\\
$j=15$	&	$t_{15}=$ May 12	&	$n_{15}=1200$	&	$P_{1200}^{15}(x_1,x_2,x_3) = \frac{1200!}{x_1! x_2! x_3!}0.05^{x_1}0.89^{x_2}0.06^{x_3}$	\\
$j=16$	&	$t_{16}=$ Nov 12	&	$n_{16}=1200$	&	$P_{1200}^{16}(x_1,x_2,x_3) = \frac{1200!}{x_1! x_2! x_3!}0.03^{x_1}0.92^{x_2}0.05^{x_3}$	\\
\hline 
\end{tabular}
\end{scriptsize} 
\caption{Data for probabilistic model estimation. Date, sample size and joint trinomial probability function of each survey. Using these distributions we will be able to compute their $95\%$ confidence intervals. Also, the model will be fitted with samples of these probability distributions.}
\label{c3TABLA2} 
\end{table}

\subsection{Probabilistic estimation}\label{33.2}
In this section, we are going to sample data survey for each survey, using the joint trinomial distribution set in Table \ref{c3TABLA2}. This will be done a high number of times ($10^4$ times) in order to generate a representative sample for each survey. Every time we sample data survey, we determine the model parameter estimations $\gamma_{12}$, $\gamma_{13}$, $\gamma_{23},$ using the Nelder-Mead optimization algorithm \cite{Nelder, Press} with goodness-of-fit $\chi^2$-test \cite{Groot}. The parameters with $p-$value less than $0.05$ will be rejected. The remainder will be sorted by $p-$value descending order. Selecting some of these model parameter vectors, we will be able to use the model outputs to provide a confidence band determined by the percentiles $2.5$ and $97.5$ ($95\%$ confidence interval) in each time instant. This $95\%$ model confidence band ($95\%$ MCB) is what we call \textit{probabilistic estimation}. Let us describe in detail the procedure.

\begin{enumerate}
\item Compute the quantiles $2.5$ and $97.5$ ($95\%$ CI) of each one of the joint multinomial distributions in Table \ref{c3TABLA2}, $j=1,2,\ldots,16$, for Support, Rejection and Abstention subpopulations, obtaining

\begin{eqnarray}
Q_{2.5}^{support} & = & ( 1.39, 2.08, 2.22, 2.92, 1.25, 1.25, 2.22, 0.50, 2.92, 			\nonumber \\ 
                  &   &   2.08, 2.08, 2.92, 2.92, 2.08, 3.83, 2.08 ), 						\nonumber \\
Q_{97.5}^{support} & = & ( 2.67, 4.00, 3.78, 5.17, 2.83, 2.83, 3.83, 1.58, 5.17, 		\nonumber \\ 
                   &   &   4.00, 4.00, 5.17, 5.17, 4.00, 6.25, 4.00 ),   					\nonumber \\
Q_{2.5}^{reject} & = & ( 91.80, 91.50, 91.80, 84.00, 81.90, 88.20, 88.60, 91.50, 		\nonumber \\ 
                 &   &   88.20, 87.20, 88.20, 86.20, 88.20, 87.20, 87.20, 90.40 ), 		\nonumber \\
Q_{97.5}^{reject} & = & ( 94.20, 94.40, 94.20, 87.90, 86.10, 91.70, 91.40, 94.40, 		\nonumber \\
                  &   &   91.70, 90.70, 91.70, 89.80, 91.70, 90.70, 90.70, 93.50 ), 	\nonumber \\                    
Q_{2.5}^{abstention} & = & ( 4.00, 2.92, 3.11, 8.33, 12.10, 6.50, 5.83, 4.67, 4.67, 	\nonumber \\ 
                 &   &   6.50, 5.58, 6.50, 4.67, 6.50, 4.67, 3.83 ), 						\nonumber \\
Q_{97.5}^{abstention} & = & ( 6.00, 5.17, 4.94, 11.80, 16.00, 9.58, 8.22, 7.33, 7.33,  \nonumber \\
                  &   &   9.58, 8.50, 9.58, 7.33, 9.58, 7.42, 6.25 ).  						\nonumber   
\end{eqnarray}

The $95\%$ CI determined by the above percentiles (they can be seen in Figures \ref{3bandas} and \ref{3bandas2} as vertical segments (error bars)) constitute an approximation of the survey results. Moreover, these $95\%$ CI will be valuable to find the best probabilistic estimation.

\item Let us define the following function of the parameters $\gamma_{12}$, $\gamma_{13}$ and $\gamma_{23}$: 

\begin{itemize}
\item[A)] For given values of $\gamma_{12}$, $\gamma_{13}$ and $\gamma_{23}$ parameters, compute the model output in $t_1=$ May 2005, $t_2=$ Nov 2005, ..., $t_{15}=$ May 2012 and $t_{16}=$ Nov 2012 for the three subpopulations, Support, Rejection and Abstention.
\item[B)] Compare, for each subpopulation, the model output obtained in step (2A) to the data values we will sample in step (3A) using the $\chi^2$-test and obtain a $p-$value for each subpopulation.
\item[C)] Calculate the minimum $p-$value among the three above.
\end{itemize}

\item For $i$ = $1$ to $10^4$

\begin{itemize}
\item[A)] Sample values of all the trinomial distributions in Table \ref{c3TABLA2}. Then, we will have one sample of $16$ surveys with percentages for Support, Rejection and Abstention populations from May 2005 until Nov 2012. Therefore, we will have a set of sampled data as in Table \ref{c3TABLA1}.
 
\item[B)] Find the model parameter values $\gamma_{12}^i$, $\gamma_{13}^i$ and $\gamma_{23}^i$ with the highest $p-$value (maximizing the function defined in steps (2A), (2B) and (2C)). To do that, Nelder-Mead optmization algorithm is used \cite{Nelder, Press} using as a goodness-of-fit the $\chi^2$-test.

\end{itemize}

\item Once the above process is completed, store the obtained parameter values and the $p-$value as the vector     

\[
 ( \gamma_{12}^i, \gamma_{13}^i, \gamma_{23}^i, p-\mbox{value}_i ), \ 1 \leq i \leq 10^4.
\]

\item Reject the model parameters with $p-$value less than $0.05$. In our case, $4990$ out of $10^4$ satisfy this restriction. Then, they are sorted by $p-$value descending order as follows,

\begin{equation}
 ( \gamma_{12}^i, \gamma_{13}^i, \gamma_{23}^i, p-\mbox{value}_i ), \ 1 \leq i \leq 4990. \label{3BF}
\end{equation}

\item For $k$ = $2$ to $4990$

\begin{itemize}
\item[A)] Substitute into the model the parameters $( \gamma_{12}^j, \gamma_{13}^j, \gamma_{23}^j )$, for $j=1,2,\ldots, k$, and compute the model output in $t_1=$ May 2005, $t_2=$ Nov 2005, ..., $t_{15}=$ May 2012 and $t_{16}=$ Nov 2012.  

\begin{itemize}
\item[a1)] Take the $k$ model outputs for Support, Rejection and Abstention at time instant $t_1=$ May 2005 and calculate the corresponding quantiles $2.5$ and $97.5$ ($95\%$ CI).  
\item[a2)] Take the $k$ model outputs for Support, Rejection and Abstention at time instant $t_2=$ Nov 2005 and calculate the corresponding quantiles $2.5$ and $97.5$ ($95\%$ CI).  
\item $\cdots$
\item[a16)] Take the $k$ model outputs for Support, Rejection and Abstention at time instant $t_{16}=$ Nov 2012 and calculate the corresponding quantiles $2.5$ and $97.5$ ($95\%$ CI).  
\end{itemize}

\item[B)] Now, gather the $16$ calculated quantiles $2.5$ for Support, Rejection and Abstention subpopulations and store them sequentially on the vectors $S_{2.5}^k$, $R_{2.5}^k$ and $A_{2.5}^k$, respectively.
\item[C)] Gather the $16$ calculated quantiles $97.5$ for Support, Rejection and Abstention subpopulations and store them sequentially on the vectors $S_{97.5}^k$, $R_{97.5}^k$ and $A_{97.5}^k$, respectively.

\item[D)] Calculate the $p-$values using the $\chi^2$-test to datasets obtained in steps (1), (6B) and (6C) grouped in pairs as follows,

\begin{itemize}
\item[d1)] $Q_{2.5}^{support}$ and $S_{2.5}^k$,
\item[d2)] $Q_{97.5}^{support}$ and $S_{97.5}^k$,
\item[d3)] $Q_{2.5}^{reject}$ and $R_{2.5}^k$,
\item[d4)] $Q_{97.5}^{reject}$ and $R_{97.5}^k$,
\item[d5)] $Q_{2.5}^{abstention}$ and $A_{2.5}^k$,
\item[d6)] $Q_{97.5}^{abstention}$ and $A_{97.5}^k$.
\end{itemize}

Note that, in order to know the parameter values which allow us to define the $95\%$ MCB (probabilistic estimation), we compare percentil vectors obtained by the trinomial sampling to the obtained using the model outputs considering the $4990$ optimal values.
 
\item[E)] Calculate $m_k$ the minimum $p-$value among the six above and build the pair $(k, m_k)$.

\end{itemize}

\item Select the pair $(k, m_k)$ among the $4990$ with the maximum $m_k$.

\end{enumerate}

In our case, the obtained value is $k=77$ with $m_{77}=0.972991$ and consequently the $p$-values corresponding to percentiles $2.5$ and $97.5$ for each subpopulation are greater than $m_{77}$.

Now, we take the $k=77$ set of parameters obtained in the above procedure, compute the model output from $t_1=$ May 2005 to $t_{16}=$ Nov 2012, in jumps of $0.05$ and, in each point, we calculate the percentiles $2.5$ and $97.5$ for each subpopulation ($95\%$ MCB). The result (probabilistic estimation) is depicted in Figure \ref{3bandas} as red continuous lines.

\begin{figure}[h]
 \begin{center}
  \includegraphics[scale=0.6]{IMG/3FIT.pdf}\\
  \caption{Probabilistic estimation. The vertical segments (error bars) correspond to the $95\%$ CI of the simulated data using multinomial distributions appearing in Table \ref{c3TABLA2}. The points in the middle of the segments are the mean values in Table \ref{c3TABLA1}. The continuous lines are the model $95\%$ MCB (probabilistic estimation) obtained with the described procedure. Note that most of the segments cross continuous lines determined by the model, capturing the data uncertainty. Only for Rejection and Abstention subpopulations in time instants Nov 2006 and May 2007 the uncertainty is not captured.} \label{3bandas}
\end{center}
\end{figure}   

The vertical segments (error bars) correspond to the $95\%$ CI of the survey data simulated by multinomial distributions appearing in Table \ref{c3TABLA2}. The points in the middle of the segments are the mean values collected in Table \ref{c3TABLA1}. The continuous lines are the model $95\%$ MCB obtained from the model outputs of the first $k=77$ out of $4990$ sets of model parameters that best fit samples of the multinomial distributions in Table \ref{c3TABLA2}.

\subsection{Probabilistic estimation analysis}
The idea of the probabilistic estimation described in the previous section is to obtain $95\%$ model confidence interval bands (MCB) as close as possible, in the sense of $\chi^2$-test, to $95\%$ CI of the data distributions appearing in Table \ref{c3TABLA2} (vertical segments in Figure \ref{3bandas}). This closeness depends on the model and on the data. 

Looking at the graphics in Figure \ref{3bandas}, we can see that almost all the vertical segments (error bars) cross at least a continuous line indicating that data uncertainty is captured by the model, in particular for Support subpopulation. 

Nevertheless, in Social Sciences the data may be very sensitive to punctual events and these events hardly are captured by the model. This happens if we study the Rejection and Abstention subpopulation graphics, where we can distinguish two parts. The first one, from May 2005 to May 2007, the probabilistic estimation intends to follow the data trajectory but the data uncertainty in Nov 2006 and May 2007 is not captured when sudden jumps appear. As we mentioned in Chapter \ref{ETA} and can also be seen in Figures \ref{datosS}, \ref{datosR} and \ref{datosA}, large jumps in the Rejection population correspond to large jumps in the Abstention population, in the opposite direction. We consider that the jumps in Nov 2006 and May 2007 are due to certain events that occurred from Sep 2006 to May 2007 as: increasing of vandalism acts from Sep 2006 to Dec 2006 linked with young left-wing nationalist groups; Barajas Airport Terminal 4 attack claimed by ETA (Dec 2006); in May 2007 local elections, the left-wing nationalist party EAE-ANV was allowed to present candidates in some villages and cities. In the second part, from Nov 2007 until Nov 2012, the continuous lines capture the data uncertainty.

%It is remarkable to note that the use of $\chi^2$-test in the procedure of the previous section to select the best fittings, allowed us to find $4990$ sets of model parameters for which the model estimation cannot be rejected as explanation of the data representing the studied phenomenon. In fact, we also could select the best (highest $p-$value) among all of them.  

Therefore, even though the estimation for Rejection and Abstention subpopulations do not capture the data uncertainty in two time instants, the three subpopulations capture the remainder and this leads us to consider the model and its probabilistic estimation appropriate to provide a prediction of the evolution of the population's attitude towards ETA over the next four years. 

\section{Probabilistic predictions over the next four years}\label{3.3}
Now, taking the model and the $k=77$ set of parameters obtained in the probabilistic estimation, we are going to give the probabilistic prediction over the next four years by computing the model outputs from Nov 2012 to Nov 2016 and then, obtaining the $95\%$ MCB (model continuous lines). We plot the results graphically in Figure \ref{3bandas2} and some numerical values in Table \ref{c3TABLA3}.

\begin{figure}[h]
 \begin{center}
  \includegraphics[scale=0.6]{IMG/3prediction.pdf}\\
  \caption{Probabilistic prediction. This picture is the  Figure \ref{3bandas} including the predictions over the next four years as $95\%$ MCB (model continuous lines).} \label{3bandas2}
\end{center}
\end{figure} 

\begin{table}[h]
\centering
\begin{small}
\begin{tabular}{|c|c|c|c|c|c|c|}
\hline
Date    & \multicolumn{2}{c|}{Support} & \multicolumn{2}{c|}{Rejection} & \multicolumn{2}{c|}{Abstention} 	\\
		& Mean & $95\%$ CI & Mean & $95\%$ CI & Mean & $95\%$ CI \\
\hline
May 2013 & $  3.10$ & $[  1.54,  4.32]$ & $ 90.69$ & $[ 88.02, 93.38]$ & $  6.22$ & $[  4.52,  9.01]$ \\ 
Nov 2013 & $  2.98$ & $[  1.55,  4.54]$ & $ 90.28$ & $[ 88.11, 93.15]$ & $  6.74$ & $[  4.63,  8.87]$ \\ 
May 2014 & $  2.68$ & $[  1.41,  4.06]$ & $ 90.40$ & $[ 87.87, 92.90]$ & $  6.92$ & $[  4.63,  8.78]$ \\ 
Nov 2014 & $  2.43$ & $[  1.43,  3.88]$ & $ 90.86$ & $[ 88.21, 93.55]$ & $  6.71$ & $[  4.74,  8.90]$ \\ 
May 2015 & $  2.44$ & $[  1.47,  3.84]$ & $ 91.23$ & $[ 89.03, 93.41]$ & $  6.34$ & $[  4.33,  8.66]$ \\ 
Nov 2015 & $  2.62$ & $[  1.42,  4.26]$ & $ 91.22$ & $[ 88.76, 93.25]$ & $  6.17$ & $[  4.60,  8.08]$ \\ 
May 2016 & $  2.72$ & $[  1.42,  4.17]$ & $ 91.01$ & $[ 88.56, 93.39]$ & $  6.27$ & $[  4.49,  8.99]$ \\ 
Nov 2016 & $  2.72$ & $[  1.46,  4.28]$ & $ 90.87$ & $[ 88.13, 93.13]$ & $  6.41$ & $[  4.43,  8.82]$ \\ 
\hline 
\end{tabular} 
\end{small}
\caption{Mean and $95\%$ confidence interval predictions for the coming eight Euskobarometro surveys.}
\label{c3TABLA3} 
\end{table}

Figure \ref{3bandas2} and Table \ref{c3TABLA3} show us that the attitude towards ETA of the population living in the Basque Country will remain fairly stable over the next four years.

\subsection{Robustness of the presented method}
Note that if we run the described procedure again, taking into account that the multinomial sampling is random, we may obtain a different value of $k$, however, the corresponding $m_k$ will be very similar. In fact, we did it two more times obtaining $k=129$ and $k=84$ with $m_k=0.9624$ and $m_k=0.966348$, respectively. The probabilistic estimations in these two new cases are given in Tables \ref{c3TABLA4} and \ref{c3TABLA5}. We can see that the predictions were very similar. This shows the robustness of the proposed method.

\begin{table}[h]
\centering
\begin{small}
\begin{tabular}{|c|c|c|c|c|c|c|}
\hline
Date    & \multicolumn{2}{c|}{Support} & \multicolumn{2}{c|}{Rejection} & \multicolumn{2}{c|}{Abstention} 	\\
		& Mean & $95\%$ CI & Mean & $95\%$ CI & Mean & $95\%$ CI \\
\hline
May 2013 & $  3.08$ & $[  1.48,  4.76]$ & $ 91.02$ & $[ 88.40, 93.44]$ & $  5.90$ & $[  4.42,  8.33]$ \\ 
Nov 2013 & $  3.09$ & $[  1.59,  4.61]$ & $ 90.41$ & $[ 87.90, 93.33]$ & $  6.50$ & $[  4.35,  8.97]$ \\ 
May 2014 & $  2.86$ & $[  1.39,  4.52]$ & $ 90.30$ & $[ 88.05, 92.89]$ & $  6.84$ & $[  4.43,  8.89]$ \\ 
Nov 2014 & $  2.59$ & $[  1.48,  4.22]$ & $ 90.64$ & $[ 88.50, 93.28]$ & $  6.77$ & $[  4.15,  8.71]$ \\ 
May 2015 & $  2.45$ & $[  1.46,  3.86]$ & $ 91.10$ & $[ 88.54, 93.21]$ & $  6.45$ & $[  3.89,  8.53]$ \\ 
Nov 2015 & $  2.51$ & $[  1.56,  3.93]$ & $ 91.35$ & $[ 88.98, 93.31]$ & $  6.14$ & $[  3.67,  8.41]$ \\ 
May 2016 & $  2.67$ & $[  1.47,  4.32]$ & $ 91.28$ & $[ 88.79, 93.33]$ & $  6.05$ & $[  3.56,  8.29]$ \\ 
Nov 2016 & $  2.75$ & $[  1.31,  4.40]$ & $ 91.06$ & $[ 88.09, 93.19]$ & $  6.19$ & $[  3.74,  8.75]$ \\ 
\hline 
\end{tabular} 
\end{small}
\caption{Mean and $95\%$ confidence interval predictions for the coming eight Euskobarometro surveys for the second procedure execution, $k=129$ and $m_k=0.9624$.}
\label{c3TABLA4} 
\end{table}

\begin{table}[h]
\centering
\begin{small}
\begin{tabular}{|c|c|c|c|c|c|c|}
\hline
Date    & \multicolumn{2}{c|}{Support} & \multicolumn{2}{c|}{Rejection} & \multicolumn{2}{c|}{Abstention} 	\\
		& Mean & $95\%$ CI & Mean & $95\%$ CI & Mean & $95\%$ CI \\
\hline
May 2013 & $  3.13$ & $[  1.73,  4.79]$ & $ 90.89$ & $[ 88.29, 93.22]$ & $  5.98$ & $[  4.47,  8.32]$ \\ 
Nov 2013 & $  3.08$ & $[  1.79,  4.59]$ & $ 90.34$ & $[ 87.87, 92.99]$ & $  6.57$ & $[  4.38,  8.81]$ \\ 
May 2014 & $  2.85$ & $[  1.47,  4.61]$ & $ 90.31$ & $[ 88.17, 92.28]$ & $  6.84$ & $[  4.79,  8.89]$ \\ 
Nov 2014 & $  2.56$ & $[  1.47,  4.23]$ & $ 90.66$ & $[ 88.11, 92.88]$ & $  6.78$ & $[  4.79,  8.67]$ \\ 
May 2015 & $  2.39$ & $[  1.50,  3.66]$ & $ 91.12$ & $[ 88.65, 93.34]$ & $  6.50$ & $[  4.27,  8.51]$ \\ 
Nov 2015 & $  2.48$ & $[  1.54,  3.84]$ & $ 91.34$ & $[ 89.30, 93.05]$ & $  6.18$ & $[  4.40,  8.37]$ \\ 
May 2016 & $  2.67$ & $[  1.47,  4.26]$ & $ 91.17$ & $[ 88.63, 93.08]$ & $  6.15$ & $[  4.71,  8.09]$ \\ 
Nov 2016 & $  2.71$ & $[  1.43,  4.35]$ & $ 90.94$ & $[ 87.93, 93.28]$ & $  6.35$ & $[  4.18,  8.69]$ \\ 
\hline 
\end{tabular} 
\end{small}
\caption{Mean and $95\%$ confidence interval predictions for the coming eight Euskobarometro surveys for the third procedure execution, $k=84$ and $m_k=0.966348$.}
\label{c3TABLA5} 
\end{table}

Also, we should say that last June 27th, 2013 was published the Euskobarometro of May 2013 with $1200$ interviews and values given in Table \ref{c3TABLA6}. The $95\%$ confidence intervals of this last Euskobarometro were calculated as in the Step 1 of the procedure described in Section \ref{33.2}.

\begin{table}[h]
\centering
\begin{small}
\begin{tabular}{|c|c|c|c|c|c|c|}
\hline
Date    & \multicolumn{2}{c|}{Support} & \multicolumn{2}{c|}{Rejection} & \multicolumn{2}{c|}{Abstention} 	\\
		& Mean & $95\%$ CI & Mean & $95\%$ CI & Mean & $95\%$ CI \\
\hline
May 2013 & $3$ & $[ 2.08,  4.00]$ & $89$ & $[ 87.17, 90.75]$ & $8$ & $[  6.50,  9.58]$ \\ 
\hline 
\end{tabular} 
\end{small}
\caption{Mean and $95\%$ confidence interval of the Euskobarometro corresponding to May 2013.}
\label{c3TABLA6} 
\end{table}

Comparing data in Table \ref{c3TABLA6} to results in Tables \ref{c3TABLA3}, \ref{c3TABLA4} and \ref{c3TABLA5}, we can see that the data uncertainty in Euskobarometro May 2013 is captured by our predictions in the three tables.

%\section{Simulating policies to reduce the Supporters population}\label{3.4}
%We are assuming that the Supporters population is the source of ETA members (Chapter \ref{ETA}) and, as we see in the previous section, with the current policies the Supporters population will remain fairly stable over the next four years. If the policymakers want to reduce this population in order to accelerate ETA's extinction, they should enforce political measures to modify the parameters $\gamma_{12}$ and $\gamma_{13}$, the ones appearing in the equation (\ref{3eq1}) that influences the variation of the Supporters population.
%
%Thus, let us suppose that the policymakers enforce a law in Jan 2013 such that parameters $\gamma_{12}$ and $\gamma_{13}$ are multiplied by $1.5$ that is, we take $50\%$ more people away from Supporters population or we avoid entering $50\%$ less people to Supporters population than the current policy. We simulate this scenario and some model outputs for the dates when the coming eight Euskobarometro surveys will be published, are given in Table \ref{c3TABLA4}.
%
%\begin{table}[h]
%\centering
%\begin{small}
%\begin{tabular}{|c|c|c|c|c|c|c|}
%\hline
%Date    & \multicolumn{2}{c|}{Support} & \multicolumn{2}{c|}{Rejection} & \multicolumn{2}{c|}{Abstention} 	\\
%		& Mean & $95\%$ CI & Mean & $95\%$ CI & Mean & $95\%$ CI \\
%\hline
%May 2013 & $  2.74$ & $[  1.50,  3.95]$ & $ 89.80$ & $[ 87.03, 92.84]$ & $  7.46$ & $[  5.24, 10.33]$ \\ 
%Nov 2013 & $  1.61$ & $[  0.56,  2.97]$ & $ 90.19$ & $[ 88.08, 93.24]$ & $  8.20$ & $[  4.65, 10.17]$ \\ 
%May 2014 & $  1.08$ & $[  0.52,  2.00]$ & $ 91.99$ & $[ 89.68, 94.27]$ & $  6.93$ & $[  4.16,  8.88]$ \\ 
%Nov 2014 & $  1.19$ & $[  0.68,  2.26]$ & $ 93.30$ & $[ 91.24, 95.26]$ & $  5.51$ & $[  3.79,  7.62]$ \\ 
%May 2015 & $  1.95$ & $[  0.92,  4.03]$ & $ 92.97$ & $[ 89.83, 94.67]$ & $  5.08$ & $[  3.56,  6.81]$ \\ 
%Nov 2015 & $  2.52$ & $[  1.19,  4.09]$ & $ 91.25$ & $[ 87.89, 94.06]$ & $  6.23$ & $[  3.92, 10.31]$ \\ 
%May 2016 & $  2.13$ & $[  0.64,  3.79]$ & $ 90.41$ & $[ 87.74, 93.51]$ & $  7.46$ & $[  5.09, 10.57]$ \\ 
%Nov 2016 & $  1.48$ & $[  0.54,  2.92]$ & $ 91.15$ & $[ 88.20, 94.43]$ & $  7.37$ & $[  4.79,  9.78]$ \\ 
%\hline 
%\end{tabular} 
%\end{small}
%\caption{Simulation of a policy to reduce the Supporters population. Mean and $95\%$ confidence interval predictions at the expected dates of the coming eight Euskobarometro surveys.}
%\label{c3TABLA4S} 
%\end{table}
%
%The figures in Table \ref{c3TABLA4S} indicate that, even though the effort of the new policy, at the end of 2016 we get an average reduction of around $1.5\%$ in the Supporters population (from $3\%$ in Nov 2012 to $1.48\%$ in Nov 2016). This fact links with the idea of Castillo-Ch\'avez \& Song in \cite{Fanatismo} cited in Chapter \ref{CAPINTRO} where they say "... even though the core population is on its way of extinction, it can still experience grow and expand in finite time before it begins to decay". It is clear that this finite time may be long.
%
%Other aspect we should take into account is the fact that it is possible that a minimum critical mass of Supporters must be necessary for ETA's survival. Although its number or percentage is unknown, maybe the simulated policy has achieved a Supporters percentage less than the critical mass and consequently, ETA would be in its way to die out. 

\section{Conclusion}\label{3.5}
In this chapter, it is presented a computational technique to deal with uncertainty (in parameter estimation and output predictions) in dynamic social models based on systems of differential equations. This technique takes data from surveys to introduce the uncertainty into the model from the very beginning and returns $95\%$ model confidence interval bands that capture the data uncertainty and predict what will happen in the near future. 
  
The technique is applied to study the evolution dynamics of the attitude of Basque population towards ETA using the model stated in Chapter \ref{paper2}. Thus, we determine a probabilistic estimation in order to find out if the model captures the Euskobarometro data evolution. We observe that the model captures the data uncertainty only partially from May 2005 to May 2007, but from Nov 2007 the probabilistic estimation improves perceptibly. Anyway, the model estimation is non-rejectable using the $\chi^2$-test. Then, we provide a probabilistic prediction of the attitude towards ETA of the population of the Basque Country over the next four years. Prediction figures indicate stabilization in the evolution of the attitudes towards ETA over the next few years, and therefore stabilization in a hypothetical pool of candidates willing to join the organization in upcoming years. As a result, the presented prediction states that the popular support to the ETA will remain stable, if and when the current scenario does not change.

Additionally, some benefits that can be obtained with this approach are:

\begin{itemize}
\item If we consider the model parameters as random variables, the technique presented here as probabilistic estimation allows the estimation of samples of these model parameters (Steps 3, 4 and 5 of the procedure described in Section \ref{33.2}). This fact is of paramount interest because one of main challenges in modelling real problems using random differential equations is to determine the distribution function of model parameters. Therefore, if we use the probabilistic estimation to obtain some samples of the model parameters, in our case $k=77$ parameter samples, we can use these samples and statistical hypothesis testing or kernel functions in order to find distribution functions of the model parameters. 
\item Other aspect that should be mentioned and it could be interesting for survey prediction estimations is the fact that Table \ref{c3TABLA3} (\ref{c3TABLA4} and \ref{c3TABLA5}) may be considered as an estimation of the results of the coming Euskobarometro surveys (mean and $95\%$ confidence interval). This idea may be applied to this and other type of surveys where a reliable underlying dynamic model can be built. As a consequence, some surveys may not be carried out with the corresponding saving of money. Therefore, we consider that this approach may be an interesting tool for social behavior studies.
\end{itemize}

