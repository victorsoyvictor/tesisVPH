% !TeX spellcheck = en_US
\chapter{Conclusion}\label{conclusion}
In this dissertation, we present a computational model to describe the transmission dynamics of HPV. The model is based on networks of lifetime sexual partners (LSP) determining the paths where HPV spreads.

Here, we intends to help in the current discussion about two main points: why the vaccination campaigns are more effective than expected and whether the boys should be also vaccinated.  

As a result of the work done with the above goals in mind, in the following, we point out the main general conclusions of this dissertation.

Under the Public Health point of view:
\begin{enumerate}
	\item Our model reproduces the singular situation occurred in Australia where a special vaccination strategy was carried out.
	\item Also, the model explains why the vaccination campaigns are more effective than expected and describes the herd immunity effect more accurately.
	\item The model shows that, only vaccinating girls, there is not herd immunity effect on MSM.
	\item The resilence is much less when vaccinating boys and girls. 
	\item An increasing in the number of LSP does not have a significant effect on the decline of HPV infections.
	\item An increasing in MSM has effect only on MSM.
\end{enumerate}

Therefore, if we vaccinate women, they are protected and heterosexual men are also protected by herd immunity. MSM are not protected at all.

If we want to eradicate oncogenic HPV, we should vaccinate boys and girls with high coverage, making sure that MSM are vaccinated with the highest coverage possible to avoid the HPV keeps circulating among unprotected MSM subnetworks.

We expect to give tools to those responsible for public health to be able to design appropriate strategies of vaccination combining effectiveness and economy.

Under the technical point of view:
\begin{enumerate}
	\item We have designed an epidemiological model to study the transmission dynamics of HPV using LSP networks.
	\item We use known real data to build big LSP networks (in networks, size may matters, see Figure 5 in \cite{villanueva2013epidemic}.
	\item We have designed a complex and innovative system to calibrate big network models.
	\item We have designed algorithms for model calibration taking into account the uncertainty in the network model building and in the data. 
	\item We have provided some advances in the computational treatment of the uncertainty quantification in computation demanding models.
\end{enumerate}
