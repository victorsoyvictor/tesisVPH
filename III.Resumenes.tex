
\chapter*{Abstract}
Sexually transmitted diseases (STDs) have been a major public health threat for a long time in human history. Modern concerns about STD began with the pandemic of syphilis which spread over Europe in the early sixteenth century. 

The human papillomavirus (HPV) is the direct cause of more than half million new cases of cervical cancer, the second most common malignancy among women and a leading cause of cancer death worldwide. It also causes anogenital warts and other related diseases.

In this work we have studied the transmission dynamics of HPV over a sexual contacts network. In order to predict the evolution of these kind of diseases, we need a reliable model of the underlying social network in which the infection spreads. We have built a lifetime sexual partners (LSP) network based on demographic data and surveys about sexual habits.

Most of the modeling approaches to STD in general and HPV in particular, are done using classical models where the hypothesis of homogeneous mixing (everybody can transmit a disease to everybody) is assumed. However, homogeneous mixing is not a reasonable hypothesis and consequences of this assumption can be seen, for instance, in that the effects of vaccination schedules against HPV have been detected in Australia much sooner than what the classical models predicted. %\bibitem{fairley2009rapid}.

There is a debate concerning the vaccination of young men. Elbasha et al. found some evidences that the vaccination of boys could also be cost-effective. In our model we consider both heterosexual men, and men who have sex with men (MSM) populations and the connections among them letting us to study this matter. With our model simulate and carry out vaccination campaigns in order to figure out the best strategies. All these results can be useful for policy makers in Public Health to make appropriate decisions respect to HPV.

\chapter*{Resumen en Castellano}
Desde tiempos inmemorables en la historia de la humanidad las enfermedades de transmisi\'on sexual (ETSs) han sido una gran amenaza para la salud p\'ublica. Las preocupaciones comienzan en la edad moderna con pandemias tales como la s\'ifilis, cuya propagaci\'on ocurre en Europa a comienzos del siglo XVI.

El virus de papiloma humano (VPH) es la causa directa de m\'as de medio mill\'on de casos nuevos de c\'ancer de cuello de \'utero, el segundo m\'as maligno entre mujeres y una de las principales causas de muerte por c\'ancer en todo el mundo. Adem\'as causa verrugas anogenitales y otras enfermedades relacionadas.

En este trabajo estudiamos el contagio del VPH en una red de contactos sexuales. Para predecir la evoluci\'on de este tipo de enfermedades, necesitamos un modelo fiable de la red social subyacente sobre el que la infecci\'on prolifera. Hemos construido una red de parejas sexuales durante toda la vida basada en datos demogr\'aficos y encuestas sobre h\'abitos sexuales.

La mayor\'ia de los enfoques para modelizar ETSs por lo general y del VPH en particular, se hacen usando modelos cl\'asicos donde la hip\'otesis de mezcla homog\'enea (todo el mundo puede transmitir a todo el mundo) es asumida de manera impl\'icita. Sin embargo, la mezcla homog\'enea no es una hip\'otesis razonable y las consecuencias de estas suposiciones se ven de hecho, en que los efectos de los calendarios de vacunaci\'on contra el VPH se detectan en Australia mucho antes de lo que los modelos cl\'asicos predijeron.

Hay un debate sobre la conveniencia de la vacunaci\'on de los ni\~nos. Elbasha et al. encontraron evidencias de que la vacunaci\'on en ni\~nos podr\'ia llegar a ser coste-efectiva. En nuestro modelo consideramos poblaciones tanto de hombres que solo tienen relaciones con mujeres y que las tienen entre ellos, permiti\'endonos sacar conclusiones al respecto. Con nuestro modelo simulamos y llevamos a cabo campa\~nas de vacunaci´\'on de modo que podemos sacar conclusiones atendiendo a las mejores estrategias. Estos resultados pueden ayudar a los responsables de Salud P\'ublica a tomar decisiones apropiadas con respecto al VPH.

\chapter*{Resum en Valenci\`a}
Des de temps inmemorables en la hist\`oria de la humanitat les malalties de transmissi\'o sexual (MTSs) han sigut una gran amena\c{c}a per a la salut p\'ublica. Les preocupacions comencen en l'edat moderna amb pand\`emies com ara la s\'ifilis, la propagaci\'o de la qual ocorre a Europa al comen\c{c}ament del segle XVI. 

El virus de papilloma hum\`a (VPH) \'es el causant directe de m\'es de mig mili\'o de casos nous de c\`ancer de coll d'\'uter, el segon mes maligne entre dones i una de les principals causes de mort per c\`ancer en tot el m\'on. A m\'es causa berrugues anogenitales i altres malalties relacionades. 

En este treball estudiem la din\`amica de transmissi\'o del VPH en una xarxa de contactes sexuals. Per a predir l'evoluci\'o d'este tipus de malalties, necessitem un model fiable de la xarxa social subjacent sobre la qual la infecci\'o prolifera. Hem construït un xarxa de parelles sexuals durant tota la vida basada en dades demogr\`afiques i enquestes sobre h\`abits sexuals.

La majoria dels enfocaments per a modelizar MTSs generalment i del VPH en particular, es fan usant models cl\`assics on la hip\`otesi de mescla homog\`enia (tot el m\'on pot transmetre a tot el m\'on) \'es assumida de manera impl\'icita. No obstant aix\`o la mescla homog\`enia no \'es una hip\`otesi raonable i les conseq\"u\`encies d'estes suposicions es veuen de fet, en que els efectes dels calendaris de vacunaci\'o contra el VPH es detecten a Austr\`alia molt abans del que els models cl\`assics van predir.

Hi ha un debat en el que referix a la vacunaci\'o dels xiquets. Elbasha et al. van trobar evid\`encies que la vacunaci\'o en xiquets podria arribar a ser cost-efectiva. En el nostre model considerem poblacions tant d'h\`omens que tenen realcions soles amb dones i els que tamb\'e tenen relacions amb homes i les connexions existents entre ells ens permeten traure conclusions sobre este aspecte. Amb el nostre model podem simular diverses campanyes de vacunaci\'o de manera que podem traure conclusions atenent a les millors estrat\`egies. Estos resultats poden ajudar als responsables de Salut P\'ublica a pendre decissions apropiades respecte al VPH.
